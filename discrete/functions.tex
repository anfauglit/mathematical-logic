\documentclass[letterpaper, 10pt]{article}
\usepackage{amsfonts,amsmath,amssymb, amsthm}
\usepackage{mathtools}
\usepackage{enumitem}

\setlength{\parindent}{0pt}

\theoremstyle{definition}
\newtheorem{thm}{Theorem}[section]
\newtheorem{cor}[thm]{Corollary}
\newtheorem{lem}[thm]{Lemma}
\newtheorem{define}[thm]{Definition}
\newtheorem*{define*}{Definition}

\newcommand{\set}[1]{\{#1\}}
\newcommand{\tuple}[1]{\langle #1 \rangle}
\renewcommand{\gcd}[1]{\left( #1 \right)}
\newcommand{\abs}[1]{\left| #1 \right|}
\renewcommand{\cong}[3][n]{#2 \equiv #3(\textrm{mod}\ #1)}
\newcommand{\divides}[2]{#1 \mid #2}
\renewcommand{\vec}[1]{\mathbf #1}
\newcommand{\vect}[1]{\overrightarrow{#1}}
\newcommand{\comp}[1]{\overline{#1}}
\newcommand{\powerset}[1]{\mathcal{P}(#1)}
\renewcommand{\implies}{\Rightarrow}
\newcommand{\entails}{\vdash}
\newcommand{\bicond}{\Leftrightarrow}
\newcommand{\preorder}{\preccurlyeq}
\def\myfunc#1#2#3{#1 \colon #2 \to #3}
\def\mymap#1#2{#1 \mapsto #2}

% Sequence macros
\def\sequence#1#2#3#4{%
	\newcount\index
	\index=#2
	\loop
	#1_\the\index,
	\advance\index by1
	\ifnum\index<3
	\repeat
	\ldots,
	#1_#3
	\if#4i,\ldots
	\fi
}
\def\osequence#1#2#3#4{%
	\newcount\index
	\index=#2
	\loop
	#1_\the\index#4
	\advance\index by1
	\ifnum\index<3
	\repeat
	\cdots#4
	#1_#3
}

\def\term#1#2{%
	\frac{#2_#1}{10^#1}
}
\def\polyterm#1#2#3{%
	% 1:variable 2:coefficient 3:index
	\if#3n
		#2_{#3}#1^{#3}
		\else\ifnum#3=1
	#2_#3#1
	\else\ifnum#3=0
		#2_#3
	\else
	\fi
	\fi
	\fi
}

\def\polynomial#1#2#3{%
	%1:variable 2:coefficient 3:index
	\polyterm #1#2#3 +
	\cdots +
	\polyterm #1#2{1} +
	\polyterm #1#2{0}
}

\def\dsequence#1#2#3#4{%
	\newcount\index
	\index=#2
	\newcount\lindex
	\lindex=1
	\loop
		\term{\the\index}{#1} #4
	\advance\index by1
	\ifnum\lindex>0
	\advance\lindex by-1
	\repeat
	\cdots #4\term #3 #1
}

\begin{document}
Let $f$ denote a function from $A$ into $B$. Then we write
\[
	\myfunc fAB
\]
We can describe a function my a mathematical formula. We may write:
\[
	\mymap x{x^2}
\]

Let $A$ be any set. The function from $A$ into $A$ which assigns to each element
in $A$ the element itself is called the \emph{identity function} on $A$ and it
is usually denoted by $1_A$, or simply $1$.
Suppose $S$ is a subset of $A$, that is, suppose $S \subseteq A$. The
\emph{inclusion map} or \emph{embedding} of $S$ into $A$, denoted by
$\iota\colon S \hookrightarrow A$.

\begin{define*}
	A function $\myfunc fAB$ is a relation from $A$ to $B$ such that each $a \in
	A$ belongs to a unique ordered pair $(a,b)$ in $f$.
\end{define*}

By a \emph{real polynomial function}, we mean a function $\myfunc
f{\mathbb{R}}\mathbb{R}$.
\[
	\polynomial xan
\]
\end{document}
