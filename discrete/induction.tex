\documentclass[letterpaper, 10pt]{article}
\usepackage{amsfonts,amsmath,amssymb,amsthm}
\usepackage{mathtools}
\usepackage{enumitem}
\usepackage[T2A]{fontenc}
\usepackage[utf8x]{inputenc}
\usepackage[russian,english]{babel}
\usepackage{amstext}
\usepackage{array}

\setlength{\parindent}{0pt}

\theoremstyle{definition}
% Numbered theorems
\newtheorem{thm}{Theorem}[section]
\newtheorem{cor}[thm]{Corollary}
\newtheorem{lem}[thm]{Lemma}
\newtheorem{define}[thm]{Definition}
% Unnumbered theorems
\newtheorem*{thm*}{Theorem}
\newtheorem*{define*}{Definition}
\newtheorem*{lem*}{Lemma}
\def\set#1{\left\{#1\right\}}
\newcommand{\tuple}[1]{\left< #1 \right>}
%\renewcommand{\gcd}[1]{\left( #1 \right)}
\def\abs#1{\left\vert #1 \right\vert}
\renewcommand{\cong}[3][n]{#2 \equiv #3(\textrm{mod}\ #1)}
\newcommand{\divides}[2]{#1 \mid #2}
\chardef\divideOp=`\|
\renewcommand{\vec}[1]{\mathbf #1}
\newcommand{\vect}[1]{\overrightarrow{#1}}
\newcommand{\setcomplement}[1]{\overline{#1}}
\let\boolneg\setcomplement
\newcommand{\powerset}[1]{\mathcal{P}(#1)}
\newcommand{\entails}{\vdash}
\newcommand{\follows}{\vDash}
\def\myfunc#1#2#3{#1 \colon #2 \to #3}
\let\implies\rightarrow
\let\iff\leftrightarrow
\let\equivx\Longleftrightarrow
\let\impliesx\Longrightarrow

\def\setdiff{\backslash}
\def\forallx#1{\forall #1\,}
\def\existsx#1{\exists #1\,}

% Symbols redefenitions
\let\preorder\preccurlyeq
\let\emptyset\varnothing
\let\le\leqslant
\let\ge\geqslant
\let\epsilon\varepsilon

% Sequence macros

% A regular sequence indexed by integers with a comma as a delimiter for
% terms.

\def\finitesequence#1#2#3#4#5{%
	%1:variable 2:first index 3:last index 4:number of initial terms
	%5:infinity flag
	\newcount\index
	\index=#2
	\advance\index by -1 % After executing the body of the loop the value of index
	% will be equal to the number of terms already printed
	\loop
		\advance\index by 1 
		#1_\number\index,
	\ifnum\index<#4
	\repeat
	\ldots, #1_#3
	\if#5i,\ldots
	\fi
}

% Finite sequence generator
\def\makefiniteseq#1#2#3#4{%
    %1:sequence name and the name of a sequence variable
    %2:first index 3:last index 4:number of initial terms 5:infinity flag
    \expandafter\def\csname seq#1\endcsname{\finitesequence #1#2#3#4n}
}

\def\rseq#1{%1:sequence name 2:sequence variable
    \csname seq#1\endcsname
}
%
%\def\osequence#1#2#3#4#5{%
%        %1:variable 2:first index 3:last index 4:delimiter
%		%5:number of initial terms
%	\newcount\index
%	\index #2
%	\advance\index -1
%	%\newcount\loopindex
%	%\loopindex #5
%	\loop
%		\advance\index 1
%		#1_\the\index#4
%        %\advance\loopindex by-1
%	%\ifnum\loopindex>0
%	\ifnum\index<#5
%	\repeat
%	\cdots#4
%	#1_#3
%}
%
%\def\makesequence#1#2#3#4#5{%
%    %1:name
%    %2:starting index 3:last index 4:delimiter
%    \expandafter\def\csname seq#1\endcsname##1{\osequence ##1#2#3#4#5}
%}
%\def\myseq#1#2{%1:sequence name 2:sequence variable
%    \csname seq#1\endcsname{#2}
%}
%
%\def\sequence#1#2{%1:first element 2:number of elements
%    \newcount\n
%    \newcount\loopindex
%    \loopindex=#2
%    \n=#1
%    \loop
%    \the\n
%    \advance\n by 1
%    \advance\loopindex by -1
%    \ifnum\loopindex>0
%    ,
%    \repeat
%}
%
%\def\dsequence#1#2#3{%1:first element 2:number of elements
%    \newcount\n
%    \newcount\loopindex
%    \loopindex=#2
%    \n=#1
%    \loop
%    \myterm{#3}{\the\n}
%    \if\the\n0
%    ,
%    \fi
%    \advance\n by 1
%    \advance\loopindex by -1
%    \ifnum\loopindex=0
%    \ldots
%    \fi
%    \ifnum\loopindex>0
%    \repeat
%}
%\def\baseterm#1#2{%
%	% 1 := variable
%	% 2 := index
%	#1_#2
%}
%
%\def\modterm#1#2#3{%
%	% 1 := variable
%	% 2 := index
%	% 3 := common term of a sequence
%	\csname #3\endcsname{#1}{#2}
%	}
%
%\def\abso#1#2{%
%	\abs{\baseterm{#1}{#2}}
%	}
%
%\def\base#1#2{%
%	\baseterm{#1}{#2}
%	}
%
%\def\reciprocal#1#2{%
%	\ifcat#21
%	\ifnum#2=1
%	1
%	\else\frac{1}{#2}
%	\fi
%	\else\frac{1}{#2}
%	\fi
%}
%
%\def\rise#1 by#2{%
%	\newcount\i
%	\newcount\n
%	\i=#2
%	\n=#1
%	\loop
%	\advance\i by -1
%	\ifnum\i>0
%	\multiply\n by#1
%	\repeat
%	\number\n
%}
%
%\def\decimal#1#2{%
%	\ifx#2n
%	\frac{1}{10^n}
%	\else
%	\newcount\demon
%	%\demon={\rise10 by2}
%	%\number\demon
%	%\demon=\rise10 by2
%	%\number\demon
%	%\frac{1}{\number\demon}
%	\fi
%}
%
%\def\reciprocalx#1#2{%
%	\ifx#2n
%	\frac{#2}{#2 + 1}
%	\else
%	\newcount\denom
%	\denom=#2
%	\advance\denom by1
%	\frac{#2}{\number\denom}
%	\fi
%}
%
%\def\fterm#1#2{%
%	f_#2(#1)
%	}
%
%\def\absseq#1#2#3#4{%
%	% 1 := variable
%	% 2 := index
%	% 3 := common term of a sequence
%	% 4 := operator
%	\newcount\index
%	\index=#2
%	\newcount\loopindex
%	\loopindex=2
%	\loop
%		\modterm{#1}{\number\index}{#3} #4
%	\advance\index by1
%	\ifnum\loopindex>0
%	\advance\loopindex by-1
%	\repeat
%	\cdots #4 \modterm{#1}{n}{#3} 
%}
%
%\def\dsequence#1#2#3#4{%
%	\newcount\index
%	\index=#2
%	\newcount\lindex
%	\lindex=1
%	\loop
%		\term{\the\index}{#1} #4
%	\advance\index by1
%	\ifnum\lindex>0
%	\advance\lindex by-1
%	\repeat
%	\cdots #4\term #3 #1
%}
%
%\def\term#1#2{%1:variable 2:index
%    #1_{#2}
%}
%
%\def\maketerm#1#2{%1:variable
%    \expandafter\def\csname myterm#2\endcsname##1{#1_##1}}
%    
%\def\myterm#1#2{\csname myterm#1\endcsname#2}


% Print an arithmetic sequence
\def\mysign#1{%
	\ifodd#1 + 
	\else -
	\fi
}

\def\tseq#1#2#3{
% 1:= the first term
% 2:= total number of terms
% 3:= the difference
    \newcount\i
    \newcount\loopindex
    \loopindex=0
    \i=#1
    \loop
    \advance\loopindex by 1
    \number\i
    \advance\i by #3 
    \ifnum\loopindex<#2
   	+ 
    \repeat
}

\def\vseq#1#2#3{
% 1:= the first term
% 2:= total number of terms
% 3:= the difference
    \newcount\i
    \newcount\loopindex
    \loopindex=0
    \i=#1
    \loop
    \advance\loopindex by 1
    \number\i
    \advance\i by #3 
    \ifnum\loopindex<#2
   	\mysign\loopindex 
    \repeat
}

\begin{document}
\section{Recursion}
\subsection{Recursively Defined Sets and Structures}
	Recursive definitions of sets have two parts: a basis step and a recursive
	step. In the basis step an initial collection of elements is specified, in
	the recursive step rules for formning new elements from those already known
	to be in the set are provided.
\subsection{Exercises}
\begin{enumerate}
	\item Proof that every chessboard of the size $2^n\times 2^n$, where $n$ is a
		natural number, with one square removed, can be tiled with L-trominoes.
\end{enumerate}

\section{Generalized Induction and Recursion}

t((Induction principle)
Suppose $P$ is some property of natural numbers, such that $P(0)$ is true, and
for every natural number $n$, if $P(n)$ is true then so is $P(n+1)$. Then $P$
holds of every number.
t)

t((The least element principle)
Suppose $P$ is true for some natural number. Then there is a smallest natural
number for which $P$ is true.
t)

t(
Let $A$ be any subset of the natural numbers, with the property that $0$ is an
element of $A$, and whenever some number $n$ is an element of $A$, so is $n +
1$. Then $A = \mathbb{N}$.
t)

t((Recursion theorem)
Suppose $a$ is an element of a set $A$, and $h$ is a function from $A$ to $A$.
Then there is a unique function $g \colon \mathbb{N} \to A$, having the
properties
\begin{align*}
	g(0) &= a \\
	g(n + 1) &= h(g(n)) \text{ for every natural number $n$} \\
\end{align*}
t)

\subsection{Inductively defined sets}
To inductively define a set, there are two necessary conditions: we need to have a
universal set which will be a superset for the set we are defining, and and we
need a function or functions defined on universal set that we will use in
the definition.
To build something we need to have initial building blocks and methods to
combine them into complex structures.
In the set theory when we construct the set of natural numbers we start with the
empty set and one operation on sets, namely the successor operator, i.e. we have
one initial object and one function to construct objects from the ones we have
already constructed.

The set of natural numbers is the smallest set containing $0$ and closed under
successor operation. A set is called inductive if it contains $0$ and is closed
under the successor operation.
$$
\mathbb{N}^*=\bigcap\set{I \mid I \text{ is an inductive subset of } U}
$$

\begin{enumerate}
	\item $0 \in \mathbb{N}$;
	\item If $n \in \mathbb{N}$, then $n + 1 \in \mathbb{N}$ (closure property);
	\item Nothing else is in $\mathbb{N}$ (minimality property).
\end{enumerate}

Another way of characterizing the set of natural numbers is to say that the
natural numbers are the things you can get to, starting from $0$ and applying
the successor function at most finitely many times. The set of natural numbers
constitute of objects for which there is a formation (or construction) sequence,
such that each term in a sequence is either $0$ or an application of the
successor function to one of the previous terms in the sequence.

\makefiniteseq f1k2
In the more general setup, to define a set inductively we need the following:
\begin{enumerate}
	\item an underlying universe, $U$;
	\item a set of initial elements, $B \subseteq U$;
	\item a set of functions on $U$, $\rseq f$ of various arities.
\end{enumerate}

We want to define the set $C$ of objects generated from $B$ by the functions
$f_i$.

\makefiniteseq a1m1
d(
A subset $A$ of $U$ is inductive if $B$ is a subset of $A$, and whenever $\rseq
a$ is in $A$ and $f_j$ is an $m$-ary function, then $f_j(\rseq a)$ is in $A$ as
well.
d)

$$
C^* = \bigcap\set{I \mid I \text{ is an inductive subset of } U}
$$

l(
$C^*$ is inductive.
l)

\makefiniteseq a0k2
d(
A finite sequence $\tuple{\rseq a}$ of elements of $U$ is a formation sequence
(or construction sequence) if, for each $i$, either $a_i$ is an element of $B$,
or there is a function $f_l$ and number $j_1,\ldots, j_m$ less than $i$, such
that $a_i = f(a_j_1, \ldots, a_j_m)$.
d)

Let $C_*$ be the set of elements $a$ in $U$ for which there is a formation
sequence ending in $a$.

l(
$C&*$ is a subset of $C_*$.
l)

l(
$C_*$ is a subset of $C^*$.
l)

Therefore, $C^* = C_*$.

l(
The principle of induction holds for $C$. In other words, if $D$ is any
inductive subset of $C$, then $D = C$.
l)

t((Induction principle for $C$)
Suppose $P$ is a property of elements of $C$, such that $P$ holds of all the
elements of $B$, and for each function $f_i$ if $P$ holds of each of the
elements $a_1,\ldots,a_k$, then $P$ holds of $f_i(a_i,\ldots,a_k)$. Then $P$
holds of every element of $C$.
t)
\end{document}
