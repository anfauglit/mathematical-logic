\documentclass[letterpaper, 10pt]{article}
\usepackage{amsfonts,amsmath,amssymb, amsthm}
\usepackage{enumitem}

\setlength{\parindent}{0pt}

\theoremstyle{definition}
\newtheorem{thm}{Theorem}[section]
\newtheorem{cor}[thm]{Corollary}
\newtheorem{lem}[thm]{Lemma}
\newtheorem{define}[thm]{Definition}

\newcommand{\set}[1]{\{#1\}}
\newcommand{\tuple}[2]{\langle #1, #2 \rangle}
\newcommand{\divides}{\mid} 
\newcommand{\ndivides}{\nmid} 
\newcommand{\abs}[1]{\left| #1 \right|}

\begin{document}
	\section{Equinumerosity}
	Comparing finite sets, elements of which can be enumerated usnig natural
	numbers, is easy: assign a number to each set and then compare the resulting
	numbers. An infinite set is always bigger than a finite set.
	Question: How to compare two infinite sets?
	\begin{define}
		A set $A$ is equinumerous to a set $B$ (written $A\simeq B$) if and only
		if there is one-to-one function from $A$ onto $B$.
	\end{define}

	For any $n \in \mathbb{Z}^+$, we denote by $\[n\]$ the set
	$\set{1,\ldots,n}$. We say that the set $\[n\]$ has $n$ elements.

	An arbitrary set $S$ has $n$ elements if and only if there exists a
	bijection $\iota \colon S \to \[n\]$.

	\begin{thm}
		The set $\mathbb{Z}^+$ is infinite.
	\end{thm}

	\begin{thm}
		A subset of a finite set is finite.
	\end{thm}

\end{document}
