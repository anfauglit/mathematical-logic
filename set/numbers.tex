\documentclass[letterpaper, 10pt]{article}
\usepackage{amsfonts,amsmath,amssymb, amsthm}

\setlength{\parindent}{0pt}

\newtheorem{thm}{Theorem}[section]
\newtheorem{cor}[thm]{Corollary}
\newtheorem{lem}[thm]{Lemma}
\newtheorem{define}[thm]{Definition}

\newcommand{\set}[1]{\{#1\}}
\newcommand{\tuple}[2]{\langle #1, #2 \rangle}
\newcommand{\divides}{\mid} 
\newcommand{\ndivides}{\nmid} 
\newcommand{\abs}[1]{\left| #1 \right|}

\begin{document}
	\section{The theory of rational numbers}
	We arrive at the rational numbers by generalising an idea of a number,
	starting with the set of natural numbers, $\mathbb{N}$.
	Natural numbers are closed under operations of addition and
	multiplication, but subtraction and division are not always defined.
	So, we have to extend natural numbers to the set of integers
	$\mathbb{Z}$ which is closed under subtraction, and then extend integers to
	the set of rational numbers $\mathbb{Q}$ which is closed under division by a
	nonzero.

	Alternatively, we axiomatically define a set of rational numbers and which
	includes natural numbers and integers. We specify integers as a certain
	proper subset of rational numbers.

	\subsection{Axioms of the theory}

	Rationals numbers are closed under the operations of addition and
	multiplication.
	\begin{align*}
		a + b &= b + a \quad \text{Commutativity} \\
		(a + b) + c &= a + (b + c) \quad \text{Associativity} \\
		a + 0 &= 0 \quad \text{Additive Identity Element} \\
		b + x & = a \quad \text{Diffirence} \\
	\end{align*}

	$\mathbb{N} \subset \mathbb{Q}$
	\begin{define}
		A rational number $m$ is an integer if and only $m = a - b$ for some
		naturals numbers $a$ and $b$.
	\end{define}
	
	\begin{align*}
		ab &= ba &&\quad \text{Commutativity}\\
		(ab)c &= a(bc) &&\quad \text{Associativity}\\
		a(b+c) &= ab + ac &&\quad \text{Distributivity
		of multiplication over addition}\\
		a \cdot 1 &= a &&\quad \text{Multiplicative Identity Element}\\
		bx &= a &&\quad \text{Quotient}
	\end{align*}

	\begin{define}
		A rational number $c$ is a quotient of two rational numbers $a$ and $b
		\neq 0$, denoted $\frac{a}{b}$ if and only $bc = a$.
	\end{define}

	Each rational number can be respresented as quotient of two integer numbers.
	
	Rational numbers are partitioned into positive, zero and negatives numbers.
	Positive numbers are rationals which can be represented as a quotient of two
	natural numbers, negative numbers are nonzero rationals which are not
	positive.
	\subsection{Properties of rational numbers}
	
	\begin{define}
		\begin{itemize}
			\item Denote $-a$ the solution to the equation $a + x = 0$, i.e. $-a = 0
		- a$. We can call that number the inverse of $a$.
			\item Denote $\frac{1}{a}$ the solution to the equation $ax = 1$.
		\end{itemize}
	\end{define}

	\begin{gather*}
		a \cdot 0 = 0\\
		ab=0 \rightarrow a=0 \lor b =0 \\
		a + (-a) = 0 \\
		(-1)(-1) = 1 \\
		-a = (-1)\cdot a\\
		-(-a) = a \\
		a - b = a + (-b) \\
		-(a+b) = -a - b\\
		-(a-b) = b - a\\
		a(b-c) = ab - ac\\
	\end{gather*}

	Every nonnegative integer is a natural number or can be written as $-n$,
	where $n \in \mathbb{N}$.

	If $p$ and $q$ are rational numbers, then their sum and product can be
	written as 
	\begin{gather*}
		p = \frac{a}{b} \\
		q = \frac{c}{d} \\
		p + q = \frac{ad+bc}{bd}\\
		pq = \frac{ac}{bd}
	\end{gather*}

	The sum and the product of two positive rational numbers are positive.

	\begin{define}
		Let $a$ and $b$ be integers, and $b \neq 0$, then $b$ divides $a$ if and
		only if the solution to the equation $bx = a$ is an integer number.
	\end{define}

	\begin{thm}
		Let $a$ and $b$ be integers which are both divided by $m$, then their
		sum, $a + b$, and difference, $a - b$, are divided by $m$.
	\end{thm}

	This theorem can be generalized to the case of the sum of an arbitrary
	number of numbers.

	\begin{thm}
		Let $a_1,a_2,\ldots,a_n$ be integer numbers, such that for all $1\leq i
		\leq n$ $m \divides a_i$, then $m \divides \sum_{i=1}^n a_i$.
	\end{thm}

	\begin{thm}
		Let $a$ and $b$ be numbers, such that $m \divides a$ and $n \divides b$,
		then $mn \divides ab$.
	\end{thm}
	\begin{thm}
		Let $\set{a_i}$ and $\set{b_i}$ be two families of numbers, such that $b_i
		\divides a_i$ for each $1 \leq i \leq n$, then $\prod_{i=1}^n b_i
		\divides \prod_{i=1}^n a_i$.
	\end{thm}

	\begin{cor}
		\begin{itemize}
			\item If $m \divides a$, then $m^n \divides a^n$.
			\item If one of the factors of a product is divisible by $m$, then
				the product is divisible by $m$.
		\end{itemize}
	\end{cor}

	\begin{thm}
		Let $a$ and $b$ be nonzero integer numbers, then $a \divides b$ an $b
		\divides a$ if and only $\abs{a} = \abs{b}$.
	\end{thm}

	\begin{thm}[The Division Algorithm]
		For any $b \ge 0$ and $a$, there exist unique integers $q$ and $r$ with
		$0 \leq r \le b$ such that $a= qb + r$.
	\end{thm}
\end{document}
