\documentclass[a4paper,10pt]{article}
\usepackage{amsmath, amssymb, amsthm}

% Defining numbered environment
\newtheorem{thm}{Theorem}[section]
\newtheorem{lemma}[thm]{Lemma}
\newtheorem{cor}[thm]{Corollary}

\theoremstyle{definition}
\newtheorem{define}[thm]{Definition}

% Defining a new command for the "is a divisor of" relation
\newcommand{\divides}{\mid}
% Defining a command for the Greatest Common Dividor
\newcommand{\gcd}[1]{\left(#1\right)}

\begin{document}

\section{Real numbers}
\subsection{Motivation for extension of rational numbers}

You can solve linear equations and systems of linear equations using rational
numbers.
Not every monotonic bounded sequence of rational numbers converges to the limit
that is a rational number.

\begin{thm}
    \begin{enumerate}
        \item If $a \divides b$, then $a \divides bc$, for any $c$
        \item If $a \divides b$ and $b \divides c$, then $a \divides c$.
        \item If $a\divides b$ and $a \divides c$, then $a \divides (bx + cy)$ for any $x$
        and $y$.
        \item If $a$ and $b$ are positive integers such that $a \divides b$ and
            $b \divides a$, then $a = b$.
    \end{enumerate}
\end{thm}

\begin{thm}
    If $a \divides c$ and $a + b = c$, then $a \divides b$.
\end{thm}

\begin{define}
    A positive integer $p$ greater than 1 is called a prime if it has no positive
    factors other than 1 and $p$.
\end{define}

\begin{thm}
    Every composite number has a prime factor.
\end{thm}

\begin{thm}
    If $a \divides bc$ and $\gcd{a,b} = 1$, then $a \divides c$.
\end{thm}

\begin{thm}
    If $a \divides c$ and $b \divides c$, and $\gcd{a,b} = 1$, then $ab
    \divides c$.
\end{thm}

\end{document}
