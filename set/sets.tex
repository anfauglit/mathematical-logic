\documentclass[letterpaper, 10pt]{article}
\usepackage{amsfonts,amsmath,amssymb, amsthm}
\usepackage{mathtools}
\usepackage{enumitem}
\usepackage[T2A]{fontenc}
\usepackage[utf8x]{inputenc}
\usepackage[russian,english]{babel}
\usepackage{array}

\setlength{\parindent}{0pt}

\theoremstyle{definition}
% Numbered theorems
%\newtheorem{thm}{Theorem}[section]
%\newtheorem{cor}[thm]{Corollary}
%\newtheorem{lem}[thm]{Lemma}
%\newtheorem{define}[thm]{Definition}
\newtheorem{thm}{Теорема}[section]
\newtheorem{cor}[thm]{Следствие}
\newtheorem{lem}[thm]{Лемма}
\newtheorem{define}[thm]{Определение}
% Unnumbered theorems
\newtheorem*{thm*}{Theorem}
\newtheorem*{define*}{Definition}
\newtheorem*{lem*}{Lemma}
\def\set#1{\left\{#1\right\}}
\newcommand{\tuple}[1]{\langle #1 \rangle}
%\renewcommand{\gcd}[1]{\left( #1 \right)}
\def\abs#1{\left\vert #1 \right\vert}
\renewcommand{\cong}[3][n]{#2 \equiv #3(\textrm{mod}\ #1)}
\newcommand{\divides}[2]{#1 \mid #2}
\chardef\divideOp=`\|
\renewcommand{\vec}[1]{\mathbf #1}
\newcommand{\vect}[1]{\overrightarrow{#1}}
\newcommand{\setcomplement}[1]{\overline{#1}}
\newcommand{\powerset}[1]{\mathcal{P}(#1)}
\newcommand{\entails}{\vdash}
\newcommand{\follows}{\vDash}
\def\myfunc#1#2#3{#1 \colon #2 \to #3}
\let\implies=\rightarrow
\let\iff=\leftrightarrow

% Symbols redefenitions
\let\preorder\preccurlyeq
\let\emptyset\varnothing
\let\le\leqslant
\let\ge\geqslant
\let\epsilon\varepsilon

% Sequence macros

% A regular sequence indexed by integers with a comma as a delimiter for
% terms.

\def\finitesequence#1#2#3#4#5{%
	%1:variable 2:first index 3:last index 4:number of initial terms
	%5:infinity flag
	\newcount\index
	\index=#2
	\advance\index by -1 % After executing the body of the loop the value of index
	% will be equal to the number of terms already printed
	\loop
		\advance\index by 1 
		#1_\number\index,
	\ifnum\index<#4
	\repeat
	\ldots, #1_#3
	\if#5i,\ldots
	\fi
}

% Finite sequence generator
\def\makefiniteseq#1#2#3#4{%
    %1:sequence name and the name of a sequence variable
    %2:first index 3:last index 4:number of initial terms 5:infinity flag
    \expandafter\def\csname seq#1\endcsname{\finitesequence #1#2#3#4n}
}

\def\rseq#1{%1:sequence name 2:sequence variable
    \csname seq#1\endcsname
}
%
%\def\osequence#1#2#3#4#5{%
%        %1:variable 2:first index 3:last index 4:delimiter
%		%5:number of initial terms
%	\newcount\index
%	\index #2
%	\advance\index -1
%	%\newcount\loopindex
%	%\loopindex #5
%	\loop
%		\advance\index 1
%		#1_\the\index#4
%        %\advance\loopindex by-1
%	%\ifnum\loopindex>0
%	\ifnum\index<#5
%	\repeat
%	\cdots#4
%	#1_#3
%}
%
%\def\makesequence#1#2#3#4#5{%
%    %1:name
%    %2:starting index 3:last index 4:delimiter
%    \expandafter\def\csname seq#1\endcsname##1{\osequence ##1#2#3#4#5}
%}
%\def\myseq#1#2{%1:sequence name 2:sequence variable
%    \csname seq#1\endcsname{#2}
%}
%
%\def\sequence#1#2{%1:first element 2:number of elements
%    \newcount\n
%    \newcount\loopindex
%    \loopindex=#2
%    \n=#1
%    \loop
%    \the\n
%    \advance\n by 1
%    \advance\loopindex by -1
%    \ifnum\loopindex>0
%    ,
%    \repeat
%}
%
%\def\dsequence#1#2#3{%1:first element 2:number of elements
%    \newcount\n
%    \newcount\loopindex
%    \loopindex=#2
%    \n=#1
%    \loop
%    \myterm{#3}{\the\n}
%    \if\the\n0
%    ,
%    \fi
%    \advance\n by 1
%    \advance\loopindex by -1
%    \ifnum\loopindex=0
%    \ldots
%    \fi
%    \ifnum\loopindex>0
%    \repeat
%}
%\def\baseterm#1#2{%
%	% 1 := variable
%	% 2 := index
%	#1_#2
%}
%
%\def\modterm#1#2#3{%
%	% 1 := variable
%	% 2 := index
%	% 3 := common term of a sequence
%	\csname #3\endcsname{#1}{#2}
%	}
%
%\def\abso#1#2{%
%	\abs{\baseterm{#1}{#2}}
%	}
%
%\def\base#1#2{%
%	\baseterm{#1}{#2}
%	}
%
%\def\reciprocal#1#2{%
%	\ifcat#21
%	\ifnum#2=1
%	1
%	\else\frac{1}{#2}
%	\fi
%	\else\frac{1}{#2}
%	\fi
%}
%
%\def\rise#1 by#2{%
%	\newcount\i
%	\newcount\n
%	\i=#2
%	\n=#1
%	\loop
%	\advance\i by -1
%	\ifnum\i>0
%	\multiply\n by#1
%	\repeat
%	\number\n
%}
%
%\def\decimal#1#2{%
%	\ifx#2n
%	\frac{1}{10^n}
%	\else
%	\newcount\demon
%	%\demon={\rise10 by2}
%	%\number\demon
%	%\demon=\rise10 by2
%	%\number\demon
%	%\frac{1}{\number\demon}
%	\fi
%}
%
%\def\reciprocalx#1#2{%
%	\ifx#2n
%	\frac{#2}{#2 + 1}
%	\else
%	\newcount\denom
%	\denom=#2
%	\advance\denom by1
%	\frac{#2}{\number\denom}
%	\fi
%}
%
%\def\fterm#1#2{%
%	f_#2(#1)
%	}
%
%\def\absseq#1#2#3#4{%
%	% 1 := variable
%	% 2 := index
%	% 3 := common term of a sequence
%	% 4 := operator
%	\newcount\index
%	\index=#2
%	\newcount\loopindex
%	\loopindex=2
%	\loop
%		\modterm{#1}{\number\index}{#3} #4
%	\advance\index by1
%	\ifnum\loopindex>0
%	\advance\loopindex by-1
%	\repeat
%	\cdots #4 \modterm{#1}{n}{#3} 
%}
%
%\def\dsequence#1#2#3#4{%
%	\newcount\index
%	\index=#2
%	\newcount\lindex
%	\lindex=1
%	\loop
%		\term{\the\index}{#1} #4
%	\advance\index by1
%	\ifnum\lindex>0
%	\advance\lindex by-1
%	\repeat
%	\cdots #4\term #3 #1
%}
%
%\def\term#1#2{%1:variable 2:index
%    #1_{#2}
%}
%
%\def\maketerm#1#2{%1:variable
%    \expandafter\def\csname myterm#2\endcsname##1{#1_##1}}
%    
%\def\myterm#1#2{\csname myterm#1\endcsname#2}


% Print an arithmetic sequence
\def\mysign#1{%
	\ifodd#1 + 
	\else -
	\fi
}

\def\tseq#1#2#3{
% 1:= the first term
% 2:= total number of terms
% 3:= the difference
    \newcount\i
    \newcount\loopindex
    \loopindex=0
    \i=#1
    \loop
    \advance\loopindex by 1
    \number\i
    \advance\i by #3 
    \ifnum\loopindex<#2
   	+ 
    \repeat
}

\def\vseq#1#2#3{
% 1:= the first term
% 2:= total number of terms
% 3:= the difference
    \newcount\i
    \newcount\loopindex
    \loopindex=0
    \i=#1
    \loop
    \advance\loopindex by 1
    \number\i
    \advance\i by #3 
    \ifnum\loopindex<#2
   	\mysign\loopindex 
    \repeat
}

\makefiniteseq a1n2
\makefiniteseq x1n2

\begin{document}

\section{Интуитивная теория множеств}

Будем использовать "интуитивный" подход, а не аксиоматический. Основными
понятиями теории множеств являются понятия множества и отношение принадлежности
$\in$ объекта к множеству, объект при этом называют элементом множества.
Отношение принадлежности является двуместным предикатом, обозначим его $\in
(x,y)$. Если $S$ есть множество, а $x$ есть его элемент, то пишут так $x \in S$,
т.е. $x$ и $S$ удовлетворяют предикату $\in (x,S)$, а если $x$ не является
элементом множества $S$, то пишут $x \not\in S$. Отношение принадлежности
является двуместным предикатом, обозначим его $\in (x,y)$ 

\subsection{Задание множеств}

Множество $S$ может быть задано как множество истинности произвольного
предиката, например, предиката $A(x)$ заданного на множестве $M$. В этом случае
пишут так: $S = \set{x \mid A(x)}$, где $x$ есть предметная переменная
пробегающая все объекты из заданной предметной области, а $A(x)$ одноместный
предикат заданный на этой области, т.е. $S$ это множество тех и только тех
предметов, которые удовлетворяют предикату $A(x)$. Если элемент $x$ принадлежит
данному множеству $S$, то как уже говорилось ранее, мы пишем $x \in S$, что есть
одноместный предикат. Обозначим этот предикат $S(x)$, из определения задания
множества (принципа абстракции) мы можем сказать что предикаты $S(x)$ и $A(x)$
равносильны, т.е. $S(x) \Leftrightarrow A(x)$.

Если множество конечно и содержит относительно небольшое количество элементов,
то тогда его можно задать перечисли все его элементы. Например, множество
содержащее натуральные числа $1$, $2$ и $3$ обозначается так: $\set{1,2,3}$. На
самом деле это частный случай задания множетсва при помощи предиката. Если мы в
качестве предметной области выберем множество всех натуральных чисел, а в
качестве одноместного предиката возьмем предикат $A(x) \Leftrightarrow x = 1
\lor x = 2 \lor x = 3$, то $\set{1,2,3} = \set{x \mid A(x)}$.

\subsection{Отношения равенства и включения множеств}
\begin{define}
	Два множества $A$ и $B$ считаются равными, если они состоят из одних и тех
	же элементов, т.е. предмет $x \in A$, если и только если $x \in B$.
\end{define}
	Если два множества $A$ и $B$ равны, то мы пишем $A = B$, а если не равны, то
	$A \neq B$.
	Два множества будут равны если предикаты, которыми они заданы равносильны, и
	соответственно множества истинности этих предикатов будут равны.
	\begin{define}
		Множество $A$ называется подмножеством множества $B$, если каждый
		элемент множества $A$ есть элемент множества $B$. Данное отношение
		называется отношением включения и обозначается так: $A \subseteq B$.
	\end{define}
	Если множество $A$ задано предикатом $P(x)$, а множество $B$ задано
	предикатом $Q(x)$, то $A \subseteq B$ если и только если $P(x) \Rightarrow
	Q(x)$, или, что равносильно, $P^+ \subseteq Q^+$.

	Пустым множеством называется множество не содержащее ни одного элемента.
	Пустое множество может быть задано любым тождественно ложным предикатом.
	Пустое множество обозначается символом $\emptyset$. Так как любой предикат
	является следствием тождественно ложного предиката, следовательно, множество
	истинности тождественно ложного предиката включено в множество истинности
	любого предиката.

	\begin{thm}
		Для произвольных множеств $X$, $Y$ и $Z$ справедливы следующие свойства:
		\begin{enumerate}
			\item $X \subseteq X$;
			\item если $X \subseteq Y$ и $Y \subseteq Z$, то $X \subseteq Z$;
			\item если $X \subseteq Y$ и $Y \subseteq Z$, то $X = Y$;
		\end{enumerate}
	\end{thm}

	\subsection{Операции над множествами}
	
	\subsubsection{Объединение множеств}

	\begin{define}
		Рассмотрим два множества $A$ и $B$, заданные предикатами
		$P(x)$ и $Q(x)$ на множестве $M$, соответственно. Тогда объединением
		множеств $A$ и $B$, обозначается $A \cup B$, будем называть множество
		заданное предикатом $P(x) \lor Q(x)$, т.е. множество всех тех элементов
		$a$ предметной области $M$, которые будучи подставленые в предикат $P(x)
		\lor Q(x)$ превращают его в истинное высказывание $P(a) \lor Q(a)$.
	\end{define}

	\subsubsection{Пересечение множеств}

	\begin{define}
		Рассмотрим два множества $A$ и $B$, заданные предикатами $P(x)$ и $Q(x)$
		на множестве $M$, соответственно. Тогда пересечение множеств $A$ и $B$,
		обозначается $A \cap B$, будем называть множество заданное предикатом
		$P(x) \land Q(x)$, т.е. множество всех тех элементов $a$ предметной
		области $M$, которые будучи подставленые в предикат $P(x) \land Q(x)$
		превращают его в истинное высказывание $P(a) \land Q(a)$.
	\end{define}

	\subsubsection{Свойства операций пересечения и объединения}
	\begin{thm}
		\begin{enumerate}
			\item $A \cap B \subseteq A,B \subseteq A \cup B$;
		\end{enumerate}
	\end{thm}

	\begin{thm}
		Для любых подмножеств $A$, $B$ и $C$ универсального множества $U$
		выполняются следующие равенства:
		\begin{enumerate}
			\item $A \cup B = B \cup A$
			\item $A \cup (B \cup C) = (A \cup B) \cup C$
			\item $A \cup (B \cap C) = (A \cup B) \cap (A \cup C)$
			\item $A \cup \emptyset = A$
			\item $A \cup \setcomplement A = U$
			\item $A \cup U = U$
			\item $A \cup A = A$
			\item $\setcomplement{A \cup B} = \setcomplement A \cap
				\setcomplement B$
			\item $A \cup (A \cap B) = A$
		\end{enumerate}
	\end{thm}

	Два множества $A$ и $B$ не пересекаются если их пересечение есть пустое
	множество, т.е. $A \cap B = \emptyset$.

	\subsubsection{Разность множеств}
	\begin{define}
		Рассмотрим два множества $A$ и $B$, заданные предикатами $P(x)$ и $Q(x)$
		на множестве $M$, соответственно. Тогда разностью множеств $A$ и $B$,
		обозначается $A \setdiff B$, будем называть множество заданное предикатом
		$P(x) \land \lnot Q(x)$, т.е. множество всех тех элементов $a$ предметной
		области $M$, которые будучи подставленые в предикат $P(x) \land \lnot Q(x)$
		превращают его в истинное высказывание $P(a) \land \lnot Q(a)$.
	\end{define}

	\begin{define}
		Рассмотрим множествo $A$, заданное предикатом $P(x)$ на множестве $M$.
		Тогда абсолютным дополнением $A$, обозначается $\setcomplement A$, будем
		называть множество заданное предикатом $\lnot P(x)$, т.е. множество всех
		тех элементов $a$ предметной области $M$, которые будучи подставленые в
		предикат $\lnot P(x)$, превращают его в истинное высказывание
		$\lnot P(x)$ или всех тех для которых $P(a)$ есть ложное высказывание.
	\end{define}

	\subsection{Отношения}

	Упорядоченной парой называется пара элементов $x$, $y$, обозначаемая
	$\tuple{x,y}$. В интуитивной теории множеств это понятие является основным.	
	
	\begin{define}
		Две упорядоченные пары $\tuple{x_1, y_1}$ и $\tuple{x_2, y_2}$ равны,
		обозначается $\tuple{x_1, y_1} = \tuple{x_2, y_2}$, если и только если
		их соответствующие элементы равны, т.е. $x_1 = x_2$ и $y_1 = y_2$.
	\end{define}

	Понятие упорядоченной пары обобщается до упорядоченной $n$-ки (кортежа из
	$n$ элементов, упорядоченного набора длинны $n$). Объекты составляющие
	упорядоченный набор называются его компонентами или координатами.

	Упорядоченная $n$-ка может быть задана рекурсивно:
	\[
		\tuple{\rseq x} =
		\begin{cases}
			\tuple{x_1,x_2} & \text{ если } n = 2 \\
			\tuple{\tuple{x_1,\ldots,x_{n-1}},x_n} & \text{ если } n > 2
		\end{cases}
	\]

	В таком случае нам не пришлось бы расширять область основных понятий и удалось
	обойтись только одим неопределяемым понятием упорядоченной пары.

	\begin{define}
		Прямым (декартовым) произведением множеств $\rseq X$ называется множество всех
		упорядоченных наборов $\rseq x$ длинной $n$, таких что для всех $i \in
		\set{1,\ldots,n}$ верно $x_i \in X_i$.  
	\end{define}
	Прямое произведение есть множество таких наборов, координаты которого
	выбираются из соответствующих им множеств. Индексы используются для задания
	соответствия между координатами набора и множествами.

	\begin{define}
		Отношением называется подмножество декартого произведения двух множеств.
	\end{define}
	Пусть $X$ и $Y$ произвольные множества, тогда подмножество $\rho$ их
	декартого произведения будет являеться отношением, пишем $\rho \subseteq X
	\times Y$. Отношение $\rho$ называется отношением между множествами $X$ и
	$Y$, если $X = Y$ и $X \times Y = X^2$, тогда $\rho \subseteq X^2$ и говорят
	что $\rho$ есть отношение на $X$.
	Если $\tuple{x,y}\in \rho$, т.е. пара предметов $x$ и $y$ входит в отношение
	$\rho$, то пишут $x \rho y$ и говорят, что $x$ и $y$ находятся в отношении
	$\rho$.

	Пусть $\rho$ отношение между множествами $X$ и $Y$. Тогда область
	определения отношения $\rho$ называется множество элементов $x$ множества $X$,
	таких что $x$ является первой компонентой хотя бы одной пары принадлежащей
	отношению.
	\[
		D_{\rho}=\set{x\mid \exists y (y \in Y \land \tuple{x,y} \in \rho)}
	\]

	Областью значений отношения $\rho$ называется множество элементов $y$
	множества $Y$, таких что $y$ является второй компонентой хотя бы одной пары
	принадлежащей отношению.
	\[
		R_{\rho}=\set{x\mid \exists x (x \in X \land \tuple{x,y} \in \rho)}
	\]

	\subsection{Операции над отношениями}

	\begin{define}
		Пусть дано отношение $\rho \subseteq X \times Y$. Тогда обратным к нему
		отношением, обозначаемым $\rho^{-1}$, есть
		подмножество $Y \times X$, такое что $\tuple{y,x} \in \rho^{-1}
		\Leftrightarrow \tuple{x,y} \in \rho$.
	\end{define}

	\begin{define}
		Пусть даны отношения $\rho \subseteq X \times Y$ и $\phi \subseteq Y
		\times Z$. Тогда композицией (суперпозицией, произведением) отношений
		$\rho$ и $\phi$ называется новое отношение, обозначаемое $\phi \circ
		\rho \subseteq X \times Z$,
		, такое что $\tuple{x,z} \in \phi \circ \rho
		\Leftrightarrow x\in X \land z \in Z \land \exists y(y \in Y \land 
		\tuple{x,y} \in \rho \tuple{y,z} \in \phi$.
	\end{define}

	Словами: композицией двух отношений является новое отношение состоящее из
	пар элементов, где первая компонента есть элемент области определения
	отношения $\rho$, а вторая --- элемент области значений отношения $\phi$.
	При этом выбираются только те пары для которых существует элемент-связка $y
	\in Y$. Например, элементы $x$ и $y$ связаны между собой отношением $\rho$,
	а элементы $y$ и $z$ связаны отношением $\phi$, мы можем определить новую
	связь, которая свяжет элемент $x$, связанный отношением $\rho$ с $y$, с теми
	элементами с которыми связан $y$ отношением $\phi$.
	
	\subsubsection{Классификация отношений}

	\begin{define}
		Отношение $\rho \subseteq X \times Y$ называется рефлексивным если для
		всех $x \in X$ выполняется $\tuple{x,x} \in \rho$.
	\end{define}

	\begin{define}
		Отношение $\rho \subseteq X \times Y$ называется симметричным если 
		для любой пары $\tuple{x,y} \in \rho$ выполняется $\tuple{x,y} \in \rho
		\Rightarrow \tuple{y,x} \in \rho$.
	\end{define}

	\begin{define}
		Отношение $\rho \subseteq X \times Y$ называется транзитивным если 
		для любых пар $\tuple{x,y} \in \rho$ и $\tuple{y,z} \in \rho$ выполняется 
		$\tuple{x,z} \in \rho$.
	\end{define}

	\begin{define}
		Отношение $\rho \subseteq M^2$ называется антисимметричным если 
		для любой пары элементов $x,y \in M$ выполняется $x = y \implies
		\tuple{x,y} \not\in \rho \lor \tuple{y,x} \not\in \rho$.
	\end{define}

	\subsubsection{Свойства операций над отношениями}
	\[
	\begin{array}{lrr}
		1. & (\rho^{-1})^{-1}=\rho & \text{} \\	
	\end{array}
\]
\end{document}
