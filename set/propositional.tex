Высказывание есть мыслительная форма, которая может быть представлена
повествовательным, связанным предложением, о котором имеет смысл говорить что
оно истинно или ложно.
В логике высказываний понятие высказывания относят к основным неопределеямым
понятиям. Высказывание обладает единственным признаком --- связанным с ним
значением истинности.
На самом деле проблема какие предложения естественного языка считать
высказываниями, а какие нет, не относится к проблемной области логики
высказываний.
Необходимо развивать в себе способность различать математические объекты и их
интерпретации; формальные определения теории и неформальные пояснения и
обоснования; основные понятия и утверждения (аксиомы) теории и определения и
утверждения (теоремы) выведенные дедуктивно из первых.
