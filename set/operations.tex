\documentclass[letterpaper, 10pt]{article}
\usepackage{amsfonts,amsmath,amssymb, amsthm}

\setlength{\parindent}{0pt}

\newtheorem{thm}{Theorem}[section]
\newtheorem{cor}[thm]{Corollary}
\newtheorem{lem}[thm]{Lemma}
\newtheorem{define}[thm]{Definition}

\newcommand{\set}[1]{\{#1\}}
\newcommand{\tuple}[1]{\langle #1 \rangle}
\renewcommand{\gcd}[1]{\left( #1 \right)}
\newcommand{\abs}[1]{\left| #1 \right|}
\renewcommand{\cong}[3]{#1 \equiv #2 (mod\ #3)}
\newcommand{\divides}[2]{#1 \mid #2}
\renewcommand{\vec}[1]{\mathbf #1}
\newcommand{\vect}[1]{\overrightarrow{#1}}
\newcommand{\family}[1]{\mathfrak{#1}}
\newcommand{\powerset}[1]{\mathcal{P}(#1)}

\begin{document}
	\subsection{The Algebra of Sets}
	Let $A$, $B$ and $C$ be any subsets of a universal set $U$, that is, $A,B,C
	\in \powerset{U}$. The set $\powerset{U}$ is closed under operations of
	union, intersection and complementation.
	The following equations are identities.
	\begin{enumerate}
		\item $A\cup(B\cup C) = (A\cup B)\cup C$. Associative law.
		\item $A\cup B = B\cup A$. Commutative law.
		\item $A\cup(B\cap C) = (A\cup B)\cap(A\cup C)$. Distributive law.
		\item $A\cup \varnothing = A$. The property of the empty set.
		\item $A\cup A' = U$. Complementation law.
	\end{enumerate}

	\begin{enumerate}
		\item If $A\cup B = U$ and $A\cap B = \varnothing$, then $B =
			\overline{A}$. The uniqueness of the complement.
		\item $\overline{\overline{A}} = A$. Double negation law.
		\item $\overline{\varnothing} = U$.
		\item $A \cup A = A$. Idempotent law.
		\item $A \cup U = U$.
		\item $A \cup (A \cap B) = A$. Absorption law.
		\item $\overline{A \cup B} = \overline{A} \cap \overline{B}$. DeMorgan's
			law.
	\end{enumerate}

	\subsubsection{Order}
	The following statements about sets $A$ and $B$ are equivalent to one
	another.
	\begin{enumerate}
		\item $A \subseteq B$.
		\item $A \cap B = A$.
		\item $A \cup B = B$.
	\end{enumerate}

	\section{Generalization of union, intersection, and cartesian product}
	Let $\family{A}$ be a collection of sets. The union of $\family{A}$ is
	the set, symbolized by $\bigcup\family{A}$ or $\bigcup\set{X\mid X \in
	\family{A}}$ or $\bigcup_{X\in\family{A}}X$, and equal to:
	$$
	\set{x\mid x\in X \text{ for some } X\in\family{A}}
	$$

	There are two immediate consequences of the definition:
	\begin{gather}
		\bigcup\varnothing = \varnothing \\
		\bigcup\set{A} = A \\
		\bigcup\powerset{A} = A \\
	\end{gather}

	Let $\family{A}$ be a nonempty collection of sets. The intersection of
	$\family{A}$ is the set, symbolized by $\bigcap\family{A}$ or
	$\bigcap\set{X\mid X \in \family{A}}$ or $\bigcap_{X\in\family{A}}X$, and
	equal to: $$ \set{x\mid x\in X \text{ for all } X\in\family{A}} $$

	We can remove the restriction on the collection being nonempty by specifying
	the universal set $U$, and then modifying the defition of the intersection
	of the collection of the subsets of $U$.
	$$
	\bigcap\family{A} = \set{x\in U\mid x\in X \text{ for all } X\in\family{A}}
	$$
	In this case, the intersection of the empty collection is equal to the
	universal set $U$:
	$$
	\bigcap_{X\in\varnothing}X = U
	$$

	Suppose that $y$ is a function on a set $I$ into a set $Y$. Call an element
	$i\in I$ an index, $I$ itself an index set, the range of $y$ an indexed set,
	and the function $y$ itself a family.
	$$
	y = \set{\tuple{i,y_i}\in I\times Y\mid i \in I} \\
	ran(y) = \set{y_i\mid i\in I}
	$$

	A family $y$ can be also symbolized by $\set{y_i}$ where $i\in I$, or just
	$\set{y_i}$ if the domain is clear from the context.

	A sequence is a type of family where the index set is either equal to
	$\mathbb{N}$ or $\mathbb{N}_0$.

	A family $\set{A_i}$ of subsets of $U$ is a function $A$ from some set $I$
	of indices into $\powerset{U}$.
	The union of a family can be symbolized by $\bigcup\set{A_i\mid i\in I}$ or
	$\bigcup_{i\in I}A_i$ or $\bigcup_i A_i$. The union of a family is the union
	of its range.
	We are in position now to generalize distributive laws for intersection over
	union and union over intersection, DeMorgan's laws, and the relation of
	order to the operations of intersection and union.
	\subsection{Generalisation of a Cartesian product of two sets}
	
	Generalization of distribution laws.
	$$
	(\bigcup_i A_i)\cap(\bigcup_j B_j) = \bigcup_{i,j} (A_i \cap B_j)
	(\bigcap_i A_i)\cup(\bigcap_j B_j) = \bigcap_{i,j} (A_i \cup B_j)
	$$
	where an index $i,j$ means an element from an index set $I\times J$.

\end{document}
