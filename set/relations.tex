\documentclass[letterpaper, 10pt]{article}
\usepackage{amsfonts,amsmath,amssymb, amsthm}

\setlength{\parindent}{0pt}

\newtheorem{thm}{Theorem}[section]
\newtheorem{cor}[thm]{Corollary}
\newtheorem{lem}[thm]{Lemma}
\newtheorem{define}[thm]{Definition}

\newcommand{\set}[1]{\{#1\}}
\newcommand{\tuple}[1]{\langle #1\rangle}
\begin{document}
	\section{Relations}

	\begin{define}
	The set $C$ is an ordered pair, denoted by $\tuple{x}{y}$,
	if and only if 
	$C = \set{\set{x},\set{x,y}}$.
	\end{define}

	The equivalence of two ordered pairs.\\
	Generalization to ordered n-tuples.\\

	Let $A$ and $B$ be sets. The Cartesian product of $A$ and $B$, denoted by
	$A\times B$, is a set $A\times B = \set{\tuple{a,b}\mid a\in A and b\in
	B}$.
	Let $A$ and $B$ be sets, a binary relation between $A$ and $B$ is a subset
	of the Cartesian product $A\times B$.

	Finite relations can be represented by tables and directed graphs
	(digraphs).

	Let $R\subseteq A\times B$ be a relation, and $c=\tuple{a}{b}$ be an element
	of $R$, then the projection of $c$ on $A$ is $a$. $a$ is called $A$-th
	coordinate of $c$. The projection of $R$ on $A$ is the set $\set{a\in A\mid
	\tuple{a}{b}\in R \text{ for some } b\in B}$.
	In case of binary relations the projection of $R$ on $A$ is called the
	domain of $R$, denoted $dom(R)$ or $R_D$, and projection of $R$ on $B$ is
	called the range of $R$, denoted $ran(R)$ or $R_R$.

	The cut of $R$ when $x=a\in A$ is the set $\set{b\in B\mid \tuple{x}{b} \in
	R}$.
	Let $R$ be a relation on $A$, then $A/R$ is a set of all cuts $R(x)$, where
	$x\in A$. The function from $A$ to $\mathcal{P}(A)$ that maps each element
	$a$ of $A$ to its cut $R(a)$, determines unambigously the relation $R$. Any
	relation can be defined as a collection of its cuts. Most of the authors use
	the notion 'image' instead of 'cut'. We can generalize the concept of image
	to subsets of the domain of $R$. The image of a set $X\subseteq A$ is the
	set $\set{y\in B\mid \tuple{x}{y} \in R \text{ for some } x\in X}$.

	\begin{define}
	If $\rho$ is a relation, then $x\rho y$ if and only if $\tuple{x}{y} \in
	\rho$.
	\end{define}

	\begin{define}
	If $\rho$ is a relation, then the domain of $\rho$ is the set
	$$
	\set{x \mid \exists y \tuple{x}{y} \in \rho}
	$$
	\end{define}

	\begin{define}
	If $\rho$ is a relation, then the range of $\rho$ is the set
	$$
	\set{y \mid \exists x \tuple{x}{y} \in \rho}
	$$
	\end{define}


	\begin{define}
		The set $C$ is called a cartesian product of the
		two sets $X$ and $Y$, denoted $X\times Y$,
		if and only if
		$C = \set{\tuple{x}{y}\mid x\in X \text{ and } y\in Y}$.
	\end{define}

	\begin{lem}
		Let $R$ be a relation and let $A = \bigcup\bigcup R$. Then $R\subseteq
		A \times A$.
	\end{lem}

	Let $R$ be a relation on $A$, that's it $R\subseteq A\times A$.
	Types of relations:
	\begin{itemize}
		\item Universal relation
		\item Void relation
		\item Identity relation
		\item Reflexive relation: The relation $R$ is reflexive if and only if
			for all $a\in A$ $\tuple{a}{a}\in R$.
		\item Symmetric relation: The relation $R$ is symmetric if and only if
			for all pairs of elements of $A$ $a$ and $b$ if $aRb$, then $bRa$. 
	\end{itemize}

	\subsection{Equivalence Relations}

	\begin{define}
		A relation $\rho$ in a set $X$ is an equivalence relation if and only if
		$\rho$ is reflexive, symmetric and transitive.
	\end{define}
	
	\begin{define}
		If $\rho$ is an equivalence relation on the set $X$, then a subset $A$
		of $X$ is an equivalence class if and only if there is a $x \in X$ such
		that $A = \set{y\in X\mid x\rho y }$ ($A = \rho[\set{x}]$, i.e. the
		set $A$ is an image of $\set{x}$ under $\rho$).
	\end{define}

	\begin{lem}
		Let $\rho$ and $\sigma$ be equivalence relations, then $\rho \cap
		\sigma$ is an equivalence relation.
	\end{lem}

	\begin{lem}
		Let $\rho$ be an equivalence relation on $X$ and let $Y$ be a set, then
		$\rho \cap (Y \times Y)$ is an equivalence relation on $X \cap Y$.
	\end{lem}
	\subsection{Functions}

	\begin{define}
		A relation $\rho$ is a function $f$ if and only if no two distinct
		members of $\rho$ has the same first coordinate.
		If $\tuple{x}{y} \in f$ and $\tuple{x}{z} \in f$, then $y = z$.
	\end{define}
	\begin{define}
		Let $f$ be a function, and $\tuple{x}{y}$ is an element of $f$.
		\begin{itemize}
		\item Each element of the $dom(f)$ is called an argument of $f$.
		\item Element $y$ of the $ran(f)$ is called the value of $f$ at $x$, or
			the image of $x$ under $f$.
		\end{itemize}
	\end{define}
	Each member of the domain of a function has a single relative.
	A function is a single valued relation.

	\begin{thm}
		Let $f$ and $g$ be functions. Then $f = g$ if and only if $D_f = D_g$
		and $f(x) = g(x)$ for each $x$ in the common domain.
	\end{thm}
	\begin{define}
		The set of all functions on $X$ into $Y$, symbolized $Y^X$ is a subset
		of $\mathfrak{P}(X \times Y)$.
	\end{define}
	\begin{align*}
		Y^\varnothing &= \set{\varnothing} //
		\varnothing^X &= \varnothing \text{ if } X \neq \varnothing //
	\end{align*}

\end{document}
