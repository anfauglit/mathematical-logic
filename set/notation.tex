\documentclass[letterpaper, 10pt]{article}
\usepackage{amsfonts,amsmath,amssymb, amsthm}

\setlength{\parindent}{0pt}

\newtheorem{thm}{Theorem}[section]
\newtheorem{cor}[thm]{Corollary}
\newtheorem{lem}[thm]{Lemma}
\newtheorem{define}[thm]{Definition}

\newcommand{\set}[1]{\{#1\}}
\newcommand{\tuple}[1]{\langle #1 \rangle}
\renewcommand{\gcd}[1]{\left( #1 \right)}
\newcommand{\abs}[1]{\left| #1 \right|}
\renewcommand{\cong}[3]{#1 \equiv #2 (mod\ #3)}
\newcommand{\divides}[2]{#1 \mid #2}
\renewcommand{\vec}[1]{\mathbf #1}
\newcommand{\vect}[1]{\overrightarrow{#1}}
\newcommand{\card}[1]{\lvert #1 \rvert}

\begin{document}
	\section{What is a set?}
	A set is a collection of objects called its elements. The concept of a set
	is introduced axiomatically. It is customary to designate a set by a
	capitcal Roman letter, such as $A$. The first relation symbol that we define
	is one of belonging, $\in$. We write $a\in A$ to say that an object $a$ is
	an element of the set $A$, and to say that $a$ is not an element of $A$ we
	will write $a\not\in A$.
	One of the ways to describe a set is by enumerating all of its elements.
	Certain number sets have special names reserved for them. Such sets are:
	positive (\mathbb{P}), natural (\mathbb{N}), integers (\mathbb{Z}), rational
	(\mathbb{Q}), real (\mathbb{R}), and complex (\mathbb{C}) numbers.
	Another way of describing a set is to use set-builder notation.
	$$
	A = \set{x\mid \phi(x)}
	$$
	where $\phi$ is a logical formula in which $x$ is a free variable. It is
	equalent to the following statement:
	$$
	x \in A \iff \phi(x)
	$$

	A finite set is a set which has a finite number of elements. A set that is
	not finite is infinite.
	Let $A$ be a finite set. The number of elements in a set is called its
	cardinality and denoted by $\card{A}$.
	Let $A$ be a nonempty set, $A\subseteq A$ and we call $A$ improper subset of
	$A$. All other subsets of $A$ are called proper subsets. 
	Very often you define a new set as a subset of already defined set.

	We can define two binary operations for sets, namely intersection and union.

	The universe, or universal set, is the set that includes all sets being
	discussed. Every set under discussion is a subset of the universal set.
\end{document}
