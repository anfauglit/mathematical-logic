\documentclass[letterpaper, 10pt]{article}
\usepackage{amsfonts,amsmath,amssymb, amsthm}
\usepackage{mathtools}
\usepackage{enumitem}

\setlength{\parindent}{0pt}

\theoremstyle{definition}
\newtheorem{thm}{Theorem}[section]
\newtheorem*{thm*}{Theorem}
\newtheorem{cor}[thm]{Corollary}
\newtheorem{lem}[thm]{Lemma}
\newtheorem{define}[thm]{Definition}

\theoremstyle{definition}
\newcommand{\thistheoremname}{}
\newtheorem{genericthm}[thm]{\thistheoremname}
\newenvironment{namedthm}[1]
	{\renewcommand{\thistheoremname}{#1}%
	\begin{genericthm}}
	{\end{genericthm}}

\newtheorem*{genericthm*}{\thistheoremname}
\newenvironment{namedthm*}[1]
	{\renewcommand{\thistheoremname}{#1}%
	\begin{genericthm*}}
	{\end{genericthm*}}
	

\newcommand{\set}[1]{\{#1\}}
\newcommand{\tuple}[1]{\langle #1 \rangle}
\renewcommand{\gcd}[1]{\left( #1 \right)}
\newcommand{\abs}[1]{\left| #1 \right|}
\renewcommand{\cong}[3][n]{#2 \equiv #3(\textrm{mod}\ #1)}
\newcommand{\divides}[2]{#1 \mid #2}
\renewcommand{\vec}[1]{\mathbf #1}
\newcommand{\vect}[1]{\overrightarrow{#1}}
\newcommand{\comp}[1]{\overline{#1}}
\newcommand{\powerset}[1]{\mathcal{P}(#1)}
\renewcommand{\implies}{\rightarrow}
\newcommand{\entails}{\vdash}
\newcommand{\bicond}{\leftrightarrow}
\newcommand{\preorder}{\preccurlyeq}
\renewcommand{\ge}{\geqslant}
\renewcommand{\le}{\leqslant}

\begin{document}
\section{Recursively defined sets and structural induction}

Induction is a rule of inference
\[
	\set{P(0), \forall k (P(k) \implies P(k+1))} \entails \forall n P(n)
\]
Using the two statements proven above, we can begin with $0$ and a chain of
conditional statements of the form $P(n-1) \implies P(n)$. In the last statement
$n$ will be the number we want to prove the property $P$ for. Now by starting at
$P(0)$ and continuously applying Modus Ponens we can get to $P(n)$. Since we can
choose any $n$, we can, using the method above, show $P(n)$ for any $n \in
\mathbb{N}$.
Do not forget to describe a property of natural numbers that we are going to
prove by induction.

When we consider recursively defined sets we use induction on the number of
applications of operations, or, equivalently, on the length of construction
(formation) sequence.
\end{document}
