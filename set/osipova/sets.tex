\documentclass[letterpaper, 10pt]{article}
\usepackage{amsfonts,amsmath,amssymb, amsthm}
\usepackage{mathtools}
\usepackage{enumitem}
\usepackage[T2A]{fontenc}
\usepackage[utf8x]{inputenc}
\usepackage[russian,english]{babel}

\setlength{\parindent}{0pt}

\theoremstyle{definition}
% Numbered theorems
%\newtheorem{thm}{Theorem}[section]
%\newtheorem{cor}[thm]{Corollary}
%\newtheorem{lem}[thm]{Lemma}
%\newtheorem{define}[thm]{Definition}

\newtheorem{thm}{Теорема}[section]
\newtheorem{cor}[thm]{Следствие}
\newtheorem{lem}[thm]{Лемма}
\newtheorem{define}[thm]{Определение}
% Unnumbered theorems
\newtheorem*{thm*}{Theorem}
\newtheorem*{define*}{Definition}
\newtheorem*{lem*}{Lemma}
\def\set#1{\left\{#1\right\}}
\newcommand{\tuple}[1]{\langle #1 \rangle}
%\renewcommand{\gcd}[1]{\left( #1 \right)}
\def\abs#1{\left\vert #1 \right\vert}
\renewcommand{\cong}[3][n]{#2 \equiv #3(\textrm{mod}\ #1)}
\newcommand{\divides}[2]{#1 \mid #2}
\chardef\divideOp=`\|
\renewcommand{\vec}[1]{\mathbf #1}
\newcommand{\vect}[1]{\overrightarrow{#1}}
\newcommand{\setcomplement}[1]{\overline{#1}}
\newcommand{\powerset}[1]{\mathcal{P}(#1)}
\newcommand{\entails}{\vdash}
\newcommand{\follows}{\vDash}
\def\myfunc#1#2#3{#1 \colon #2 \to #3}
\let\implies=\rightarrow
\let\iff=\leftrightarrow

% Symbols redefenitions
\let\preorder\preccurlyeq
\let\emptyset\varnothing
\let\le\leqslant
\let\ge\geqslant
\let\epsilon\varepsilon

% Sequence macros

% A regular sequence indexed by integers with a comma as a delimiter for
% terms.

\def\finitesequence#1#2#3#4#5{%
	%1:variable 2:first index 3:last index 4:number of initial terms
	%5:infinity flag
	\newcount\index
	\index=#2
	\advance\index by -1 % After executing the body of the loop the value of index
	% will be equal to the number of terms already printed
	\loop
		\advance\index by 1 
		#1_\number\index,
	\ifnum\index<#4
	\repeat
	\ldots, #1_#3
	\if#5i,\ldots
	\fi
}

% Finite sequence generator
\def\makefiniteseq#1#2#3#4{%
    %1:sequence name and the name of a sequence variable
    %2:first index 3:last index 4:number of initial terms 5:infinity flag
    \expandafter\def\csname seq#1\endcsname{\finitesequence #1#2#3#4n}
}

\def\rseq#1{%1:sequence name 2:sequence variable
    \csname seq#1\endcsname
}
%
%\def\osequence#1#2#3#4#5{%
%        %1:variable 2:first index 3:last index 4:delimiter
%		%5:number of initial terms
%	\newcount\index
%	\index #2
%	\advance\index -1
%	%\newcount\loopindex
%	%\loopindex #5
%	\loop
%		\advance\index 1
%		#1_\the\index#4
%        %\advance\loopindex by-1
%	%\ifnum\loopindex>0
%	\ifnum\index<#5
%	\repeat
%	\cdots#4
%	#1_#3
%}
%
%\def\makesequence#1#2#3#4#5{%
%    %1:name
%    %2:starting index 3:last index 4:delimiter
%    \expandafter\def\csname seq#1\endcsname##1{\osequence ##1#2#3#4#5}
%}
%\def\myseq#1#2{%1:sequence name 2:sequence variable
%    \csname seq#1\endcsname{#2}
%}
%
%\def\sequence#1#2{%1:first element 2:number of elements
%    \newcount\n
%    \newcount\loopindex
%    \loopindex=#2
%    \n=#1
%    \loop
%    \the\n
%    \advance\n by 1
%    \advance\loopindex by -1
%    \ifnum\loopindex>0
%    ,
%    \repeat
%}
%
%\def\dsequence#1#2#3{%1:first element 2:number of elements
%    \newcount\n
%    \newcount\loopindex
%    \loopindex=#2
%    \n=#1
%    \loop
%    \myterm{#3}{\the\n}
%    \if\the\n0
%    ,
%    \fi
%    \advance\n by 1
%    \advance\loopindex by -1
%    \ifnum\loopindex=0
%    \ldots
%    \fi
%    \ifnum\loopindex>0
%    \repeat
%}
%\def\baseterm#1#2{%
%	% 1 := variable
%	% 2 := index
%	#1_#2
%}
%
%\def\modterm#1#2#3{%
%	% 1 := variable
%	% 2 := index
%	% 3 := common term of a sequence
%	\csname #3\endcsname{#1}{#2}
%	}
%
%\def\abso#1#2{%
%	\abs{\baseterm{#1}{#2}}
%	}
%
%\def\base#1#2{%
%	\baseterm{#1}{#2}
%	}
%
%\def\reciprocal#1#2{%
%	\ifcat#21
%	\ifnum#2=1
%	1
%	\else\frac{1}{#2}
%	\fi
%	\else\frac{1}{#2}
%	\fi
%}
%
%\def\rise#1 by#2{%
%	\newcount\i
%	\newcount\n
%	\i=#2
%	\n=#1
%	\loop
%	\advance\i by -1
%	\ifnum\i>0
%	\multiply\n by#1
%	\repeat
%	\number\n
%}
%
%\def\decimal#1#2{%
%	\ifx#2n
%	\frac{1}{10^n}
%	\else
%	\newcount\demon
%	%\demon={\rise10 by2}
%	%\number\demon
%	%\demon=\rise10 by2
%	%\number\demon
%	%\frac{1}{\number\demon}
%	\fi
%}
%
%\def\reciprocalx#1#2{%
%	\ifx#2n
%	\frac{#2}{#2 + 1}
%	\else
%	\newcount\denom
%	\denom=#2
%	\advance\denom by1
%	\frac{#2}{\number\denom}
%	\fi
%}
%
%\def\fterm#1#2{%
%	f_#2(#1)
%	}
%
%\def\absseq#1#2#3#4{%
%	% 1 := variable
%	% 2 := index
%	% 3 := common term of a sequence
%	% 4 := operator
%	\newcount\index
%	\index=#2
%	\newcount\loopindex
%	\loopindex=2
%	\loop
%		\modterm{#1}{\number\index}{#3} #4
%	\advance\index by1
%	\ifnum\loopindex>0
%	\advance\loopindex by-1
%	\repeat
%	\cdots #4 \modterm{#1}{n}{#3} 
%}
%
%\def\dsequence#1#2#3#4{%
%	\newcount\index
%	\index=#2
%	\newcount\lindex
%	\lindex=1
%	\loop
%		\term{\the\index}{#1} #4
%	\advance\index by1
%	\ifnum\lindex>0
%	\advance\lindex by-1
%	\repeat
%	\cdots #4\term #3 #1
%}
%
%\def\term#1#2{%1:variable 2:index
%    #1_{#2}
%}
%
%\def\maketerm#1#2{%1:variable
%    \expandafter\def\csname myterm#2\endcsname##1{#1_##1}}
%    
%\def\myterm#1#2{\csname myterm#1\endcsname#2}


% Print an arithmetic sequence
\def\mysign#1{%
	\ifodd#1 + 
	\else -
	\fi
}

\def\tseq#1#2#3{
% 1:= the first term
% 2:= total number of terms
% 3:= the difference
    \newcount\i
    \newcount\loopindex
    \loopindex=0
    \i=#1
    \loop
    \advance\loopindex by 1
    \number\i
    \advance\i by #3 
    \ifnum\loopindex<#2
   	+ 
    \repeat
}

\def\vseq#1#2#3{
% 1:= the first term
% 2:= total number of terms
% 3:= the difference
    \newcount\i
    \newcount\loopindex
    \loopindex=0
    \i=#1
    \loop
    \advance\loopindex by 1
    \number\i
    \advance\i by #3 
    \ifnum\loopindex<#2
   	\mysign\loopindex 
    \repeat
}
\makefiniteseq a1n2
\makefiniteseq x1n2
\makefiniteseq X1n2

\begin{document}

\section{Множества и отношения}

\subsection{Начальные понятия теории множеств}

Под множеством понимется любая совокупность однозначно определенных и различных
между собой объектов, которая мыслится как целое. Эти объекты называются
элементами множества. Элементами множеств могут быть объекты любой природы,
множества могут быть не однородными, т.е. объединять элементы различной природы.
Отношение принадлежности, также как и понятие множества, является основным,
неопределяемым понятием теории.

Понятие принадлежности элемента множеству, не стоит мыслить как физическую
принадлежность, физическое включение, этого элемента в множество, что множество
это некий мешок в который мы можем складывать предметы --- элементы множества.
Очень быстро такая аналогия перестает работать и только мешает пониманию.
Диаграммы Эйлера-Венна являются интструментом, который также очень быстро теряет
свою адекватность. С помощью его относительно легко продемострировать
определенные отношения между множествами, а также операции, однако его
огранниченность очень быстро становиться очевидной. Например, проблемы
начинаются уже при попытке изобразить на такой диаграмме пустое множество,
область не имеющую площади изобразить на плоскости мы не можем. Соответственно
попытка показать что пересечение двух непересекающихся множеств есть пустое
множество окончится неудачей --- заштриховать фигуру с нулевой площадью не
получится. В качестве аналитических методов все эти подходы не являются хоть
сколько-нибудь перспективными и вообще говоря полезными. Их неадекватность
проявляется уже при попытке ответить на самые простые вопросы касающиеся
отношений между множествами.

Намного более полезным будет начать сразу мыслить принадлежность как отношение
между двумя объектами, где по одну сторону отношения обязательно будет стоять
множество, а по другую --- предмет любой природы. Причем чтобы побыстрей
разорвать все ассоциации с физическим включением или принадлежностью предметов
множеству, будем по возможности, вместо слова "принадлежность" использовать
слово "связь". Когда между данным объектом и множеством определена такая связь,
мы говорим, что объект является элементом множества, а множество содержит этот
объект. Множествами называются исключительно такие сущности, с которым могут
быть связаны другие объекты отношением принадлежности. Это единственное
свойство, которым обладают множества, способность связываться отношением
принадлежности с другими объектами. С множеством этой связью могут быть связаны
как многие объекты, т.е.  множество может включать в себя много объектов, и с
множеством может быть связан только один объект, такое множество мы называем
одноэлементным, однако, с множеством может быть не связано никаких объектов
вообще, такое множество мы называем пустым. 

Любое множество определяется лишь его связями (отношениями принадлежности) с
другими объектами, поэтому мы считаем два множества равными, если все их связи
совпают, т.е. оба множества связаны (состоят в отношении принадлежности) с
одними и теми же объектами.

Для того чтобы однозначно задать множество достаточно определить тот набор
предметов которые состоят с множеством в отношнении принадлежности.

В терминах отношения достаточно легко и точно можно описать операции над
множествами. Например, дадим определение операции объединения двух множеств.

Объединением двух множеств называют такое множество, которое связано (отношением
принадлежности) со всеми теми и только теми элементами с которыми связано хотя бы
одно из исходным множеств. Используя символику теории множеств и обозначив
именем $A
\cup B$ объединение двух множеств $A$ и $B$, мы может это определение переписать
следующим образом, что все те и только те предметы $x$ которые связаны с
множеством $A$, пишем $x \in A$, или множеством $B$, пишем $x \in B$, связываются
с множеством $A \cup B$, пишем $x \in A \cup B$. Предмет $x$ связан
(принадлежит) с множестом $A$ --- это есть не что иное как предикат, который мы
можем обозначить так --- $x \in A$. Предикат истинный когда рассматриваемый
предмет связан с множеством $A$ и ложен когда не связан.

Проблема возникающая при рассмотрениях операции объединения, как операции
сложения вместе содержания двух мешков в один и возможным появлением одинковых
предметов в общем мешке, просто не возникает когда мы рассматриваем связи между
предметами. В последнем случае не происходит никакого физического "копирования"
предметов, поэтому и одинаковых предметов в итоговом множеств не возникает.

Любой двуместный предикат задает определенное отношение между объектами --- два
объекта состоят в отношении если предикат для этих объектов превращается в
истинное высказывание, и не состоят, если в ложное. Отношения можно представить
в виде направленного графа. В частности отношение принадлежности объектов
множеству может быть представлено графом этого отношения.

Пересечением двух множеств будет множество связанное с теми и только тем
предметами с которыми связаны оба исходных множества. Пересечение множеств $A$ и
$B$ обозначается выражением $A \cap B$. Любой предмет $x$ должен удовлетворять
следующему предикату $x \in A \cap B \iff x \in A \land x \in B$. Словами:
предмет $x$ будет связан с множеством $A \cap B$ если он связан с обоими
множествами $A$ и $B$ и не будет связан с множеством $A \cap B$ если он не
связан хотя бы с одним из множеств $A$ и $B$. 

\begin{define}[Интуитивный принцип объемности]
	Два множества считаются равными, если они состоят из одних и тех же
	элементов.
\end{define}

В терминах предикатов, принцип объемности можно сформулировать так: $x \in A
\equivx x \in B$, т.е. два предиката равносильны.

Любое множество можно сопоставить определенному понятию: где предикат
определяющий это множетсво будет соответствовать содержанию понятия, описывающим
отличительные признаки элементов входящих во множество, а множество элементов
удовлетворяющих этому предикату, т.е. обладающих этими признаками, будет объемом
понятия.

Множество элементами которого являются объекты $\rseq a$ и только они,
обозначают $\set{\rseq{a}}$.

Формулой от $x$ называется формула логики предикатов содержащая одну свободную
переменную $x$, такая, что если каждое вхождение $x$ в эту формулу
заменить одним и тем же именем предмета из заданной предметной области, то
получится высказывание: истинное или ложное.

\begin{define}[Интуитивный принцип абстракции]
	Любая формула $P(x)$ определяет некоторое множество $A$, а именно множество
	тех и только тех предметов $a \in A$, для которых $P(a)$ есть истинное
	высказывание.
\end{define}

В целом справа от вертикальной черты может стоят любой предикат, свободными
переменными которого могут быть только переменные составляющие набор записанный
слева от вертикальной черты.

Множество предметов заданной при помощи формулы $P(x)$ обозначается $\set{x\mid
P(x)}$.

Заданием предиката с одной свободной переменной мы выделяем существенные
признаки предметов, отбрасывая все второстепенные, несущественные для данного
рассмотрения. Процесс выделения одних свойств предмета и отвлечение от других
называется абстрагированием. На основе выделенных путем абстрагирования,
заданием предиката, признаков мы может объединить предметы обладающие этими
признаками в совокупность (или множество), дав ей определенное имя (имя
множества). Такой прием, объединения в классы предметов обладающих общими
признаками называется объединением.

Когда каждый элемент множества $A$ есть элемент множества $B$, говорят что $A$
есть подмножество множества $B$, обозначается $A \subseteq B$. Такое отношение
между множествами называют отношением включения. Если при этом $A \neq B$, то
говорят что $A$ есть собственное подмножество $B$, обозначается $A \subset B$.

На языке логике предикатов определение отношения включения можно записать
следующим образом:
\[
	\subseteq (X,Y) \stackrel{\text{def}}{\equivx} \forallx x (x\in X \implies x
	\in Y)
\]

Отношение включения есть отношение порядка, так как оно обладает следующими
свойсвтвами:
\begin{enumerate}
	\item $X \subseteq X$ рефлексивность отношения включения множеств
	\item $X \subseteq Y \land Y \subseteq X \follows X = Y$ антисимметричность
	\item $X \subseteq Y \land Y \subseteq Z \follows X = Z$ транзитивность
\end{enumerate}

Множество всех подмножеств множества $A$ называется множеством-степенью (булеан) и
обозначается $\powerset{A}$.

\subsection{Операции над множествами. Алгебра множеств}

В общем случае для множеств $A$ и $B$ и универсального множества $U$, для
произвольного элемента универсального $a$ множества $U$ верно одно из следующих
утверждений: $a \in A \land a \in B$, $a \in A \land a \not\in B$, $a \not\in A
\land a \in B$, и $a \not\in A \land a \not\in B$.

Объединением множеств $A$ и $B$ называется множество $A \cup B$, все элементы
которого являются элементами множества $A$ или $B$:
$$
A \cup B = \set{x \mid x \in A \lor x \in B}
$$
Пересечением множеств $A$ и $B$ называется множество $A \cap B$, элементы
которого являются элементами обоих множеств $A$ и $B$:
$$
A \cap B = \set{x \mid x \in A \land x \in B}
$$

Операции объединения и пересечения обладают следующими свойствами:
$$
A \cap B \subseteq A,B \subseteq A \cup B
$$

Относительным дополнением множества $A$ до множества $X$ (или разностью множеств
$X$ и $A$), т.е. какие элементы множества $X$ нужно добавить к множеству $A$
чтобы $X$ было включено в $A$, или какими элементами надо пополнить $A$, чтобы
оно стало равным $A \cup X$:
$$
X \setminus A = \set{x \mid x \in X \land x \not\in A}
$$

В терминах связи множества с его элементами разностью множеств $A$ и $B$
называется такое множество с которым связаны исключительно те элементы которые
связаны (принадлежат) множеству $A$ и не связаны с множеством $B$, т.е. мы
связываем с разностью только те элементы $A$, которые не связаны одновременно с
множествами $A$ и $B$. На языке логики предикатов определение такого множества
будет записано так:
$$
x \in A \setminus B \equivx x \in A \land \boolneg{x \in B}
$$

В общем случае операция разности не коммутативна, т.е. $A \setminus B \not\eq B
\setminus A$.

Если $A \setminus B = \emptyset$, тогда $A \subseteq B$. Другими словами не
существует таких предметов, которые связаны с множеством $A$, но не связаны с
множеством $B$. Также можно сказать что множеству $A$ нечем дополнить множество
$B$.

Симметрической разностью множеств $A$ и $B$ называется множество элементов
входящих либо в множество $A$, либо в множество $B$, но не одновременно в оба
множества:
$$
X + Y = \set{x \mid (x \in X \lor x \in Y) \land x \not\in X \cap Y}
$$

Если все рассматриваемые в данной задаче множества являются подмножествами
некоторого множества $U$, то это множество называется универсальным для данной
задачи.

Абсолютным дополнением множества $A$ называется множество $\setcomplement{A}$ всех
тех элементов, которые не принадлежат множеству $A$:
$$
\setcomplement{X} = \set{x \mid x \in U \land x \not\in X}
$$

\begin{thm}
	Для любых подмножеств $A$, $B$ и $C$ универсального множества $U$
	выполняются следующие тождества:
	\begin{align*}
		A \cup B &= B \cup A & A \cap B &= B \cap A \\
		A \cup (B \cup C) &= (A \cup B) \cup C & A \cap (B \cap C) &= (A \cap B) \cap C \\
		A \cup (B \cap C) &= (A \cup B) \cap (A \cup C) & A \cap (B \cup C) &= (A \cap B) \cup (A \cap C) \\
		A \cup \emptyset &= A & A \cap U &= A \\
		A \cup \setcomplement{A} &= U & A \cap \setcomplement{A} &= \emptyset \\
		A \cup U &= U & A \cap \emptyset &= \emptyset \\
		\setcomplement{A \cup B} &= \setcomplement{A} \cap \setcomplement{B} & \setcomplement{A \cap B} &= \setcomplement{A} \cup \setcomplement{B} \\
		A \cup (A \cap B) &= A & A \cap (A \cup B) &= A \\
	\end{align*}
\end{thm}

\begin{thm}
	Следующие утверждения о произвольных множествах $A$ и $B$ попарно
	эквивалентны:
	\begin{enumerate}
		\item $A \subseteq B$
		\item $A \cap B = A$
		\item $A \cup B = B$
	\end{enumerate}
\end{thm}

\subsection{Упорядоченный набор. Прямое произведение множеств}

Упорядоченная пара $\tuple{x,y}$ есть совокупность двух элементов $x$ и $y$,
о которых мы можем сказать какой из них первый, а какой второй. Пару можно
рассматривать как индексированное множество, т.е. такое множество где каждому
элементу присвоен свой уникальный индекс, обычно в качестве индексов берется
начальный сегмент множества натуральных чисел. Две пары $\tuple{x,y}$ и $\tuple{u,v}$
считаются равными тогда и только тогда, когда $x = u$ и $y = v$. Когда мы
сравниваем две упорядоченных пары, элементы индексируемые одним и тем же
индексом должны совпадать.
Упорядоченная $n$-ка элементов $\rseq x$ обозначается $\tuple{\rseq x}$ и по
определению есть $\tuple{\tuple{x_1,x_2,\ldots,x_{n-1}},x_n}$.
Элементы $\rseq x$ называются компонентами или координатами $n$-ки.
Упорядоченная $n$-ка называется также кортежем из элементов $\rseq x$.

Прямым произведением множеств $X$ и $Y$, обозначается $X \times Y$, называется
совокупность всех упорядоченных пар $\tuple{x,y}$ таких, что $x \in X$ и $y \in
Y$.

Прямым произведением множеств $\rseq X$ называется совокупность всех
упорядоченных $n$-ок $\tuple{\rseq x}$ таких, что $x_i \in X_i, i =
1,2,\ldots,n$. Прямое произведение множеств $\rseq X$ обозначается $X_1 \times
X_2 \times \cdots X_n$. Если $X_1 = X_2 = \cdots = X_n = X$, то пишут $X_1
\times X_2 \times \cdots X_n = X^n$.

\begin{thm}
	Для произвольных множеств $A$, $B$ и $C$ верны следующие утверждения:
	\begin{align*}
		(A \cap B) \cup (A \cap \setcomplement B) &= A \\
		(A \cup B) \cap (A \cup \setcomplement B) &= A \\
		(\setcomplement A \cup B) \cap A &= A \cap B \\
		(A \setminus B) \setminus C &= (A \setminus C) \setminus (B \setminus C) \\
		A \setminus (B \cup C) &= (A \setminus B) \setminus C \\
		A \setminus (B \setminus C) &= (A \setminus B) \cup (A \cap C) \\
		A + (B + C) &= (A + B) + C \\
		A \cap (B + C) &= (A \cap B) + (A \cap C)
		(A \cup B) \subseteq \text{тогда и только тогда, когда} A \subseteq C \text{и} B \subseteq C \\
		A \subseteq B \cap C \text{тогда и только тогда, когда} A \subseteq B \text{и} A \subseteq C \\
		A \cap B \subseteq C \text{тогда и только тогда, когда} A \subseteq \setcomplement B \cup C \\
		A \subseteq B \cup C \text{тогда и только тогда, когда} A \cap \setcomplement B \subseteq C
	\end{align*}
\end{thm}
\end{document}
