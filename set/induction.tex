\documentclass[letterpaper, 10pt]{article}
\usepackage{amsfonts,amsmath,amssymb, amsthm}
\usepackage{mathtools}
\usepackage{enumitem}

\setlength{\parindent}{0pt}

\theoremstyle{definition}
\newtheorem{thm}{Theorem}[section]
\newtheorem*{thm*}{Theorem}
\newtheorem{cor}[thm]{Corollary}
\newtheorem{lem}[thm]{Lemma}
\newtheorem{define}[thm]{Definition}

\theoremstyle{definition}
\newcommand{\thistheoremname}{}
\newtheorem{genericthm}[thm]{\thistheoremname}
\newenvironment{namedthm}[1]
	{\renewcommand{\thistheoremname}{#1}%
	\begin{genericthm}}
	{\end{genericthm}}

\newtheorem*{genericthm*}{\thistheoremname}
\newenvironment{namedthm*}[1]
	{\renewcommand{\thistheoremname}{#1}%
	\begin{genericthm*}}
	{\end{genericthm*}}
	

\newcommand{\set}[1]{\{#1\}}
\newcommand{\tuple}[1]{\langle #1 \rangle}
\renewcommand{\gcd}[1]{\left( #1 \right)}
\newcommand{\abs}[1]{\left| #1 \right|}
\renewcommand{\cong}[3][n]{#2 \equiv #3(\textrm{mod}\ #1)}
\newcommand{\divides}[2]{#1 \mid #2}
\renewcommand{\vec}[1]{\mathbf #1}
\newcommand{\vect}[1]{\overrightarrow{#1}}
\newcommand{\comp}[1]{\overline{#1}}
\newcommand{\powerset}[1]{\mathcal{P}(#1)}
\renewcommand{\implies}{\Rightarrow}
\newcommand{\entails}{\vdash}
\newcommand{\bicond}{\Leftrightarrow}
\newcommand{\preorder}{\preccurlyeq}
\renewcommand{\ge}{\geqslant}
\renewcommand{\le}{\leqslant}

\begin{document}

\section{Peano Axioms}

The five Peano axioms are the following:
\begin{enumerate}[label=(\roman*)]
	\item There is a least element of $\mathbb{N}$ that we denote by $0$.
	\item Every natural number $a$ has a successor denoted by $s(a)$.
	\item There is no natural number whose successor is $0$.
	\item Distinct natural numbers have distinct successors.
	\item If a subset of the natural numbers contains $0$ and also has the
		property that whenever $a \in S$ it follows that $s(a) \in S$, then the
		subset $S$ is actually equal to $\mathbb{N}$.
\end{enumerate}

\begin{thm}
	If a proposition $P$ holds for the number $O$ and whenever $P$ holds for
	an arbitrary natural number $n$ it also holds for its successor
	$s(n)$, then $P$ holds for every natural number $n$.
\end{thm}

\begin{thm}
	To prove that a proposition $P(a)$ holds for all integers. We have to show
	that it holds for all natural numbers and for all numbers of the form $-n$,
	where $n$ is a natural number.
\end{thm}

\begin{namedthm*}{The Extended Principle of Mathematical
	Induction}\cite{fletcher}
	Let $k \in \mathbb{N}$ and let $S$ be a subset of $\mathbb{N}$ such that 
	\begin{enumerate}[label=\alph*)]
		\item $k \in S$, and 
		\item If $n \ge k$ and $n \in S$, then $n + 1 \in S$.
	\end{enumerate}
	Then $\set{n \in \mathbb{N} \mid n \ge k} \subseteq S$.
\end{namedthm*}

\begin{thm*}
	The Principle of Mathematical Induction and the Extended
	Principle of Mathematical Induction are equivalent.
\end{thm*}

\begin{namedthm*}{The Second Principle of Mathematical Induction}\cite{fletcher}
	Let $S$ be a set of natural numbers with the following properties:
	\begin{enumerate}[label=\alph*)]
		\item $1 \in S$, and 
		\item For each $n \in \mathbb{N}$, if $\set{1,2,\ldots,n} \subseteq S$,
			then $n + 1 \in S$.
	\end{enumerate}
	Then $S = \mathbb{N}$.
\end{namedthm*}

\begin{namedthm*}{The Least Natural Number Principle}\cite{fletcher}
	Every non-empty set of natural numbers has a least element.
\end{namedthm*}

\begin{namedthm*}{The Division Algorithm for Integers}\cite{fletcher}
	Let $a$ be an integer and let $b$ be a natural number. Then there are
	unique integers $q$ and $r$ such that $a = bq + r$ and $0 \le r < b$. 
\end{namedthm*}

\section*{Exercises}
Prove the following statements:
\begin{enumerate}
	\item For every natural number $n \ge 5$, $n^2 < 2^n$.
	\item For every natural number $n \ge 3$, $2n < 2^n - 1$.
	\item Every natural number greater than $1$ is either a prime number or the
		product of prime numbers.
	\item $n^3 + (n + 1)^3 + (n + 2)^3$ is divisible by $9$ for every $n \ge 1$.
	\item $3^{3n} + 1$ is divisible by $7$ whenever $n$ is an odd natural
		number.
	\item $2^n < n!$ whenever $n \ge 4$. 
	\item Each integer number greater than $1$ can be factored into a product of
		one or more primes.
	\item The binomial $1 - x^n$ can be factored as $(1-x)(1 + x + x^2 + \cdots
		+ x^{n-1})$.
	\item There are $2^n$ different binary strings of length $n$.
	\item A knight can be moved from any square to any other square on an $n
		\times n$ chess board by some sequence of allowed moves, for every $n
		\ge 4$.
	\item A truth table involving $n$ statement variables requires $2^n$ rows.
\end{enumerate}

\bibliographystyle{plain}
\bibliography{refs}
\end{document}
