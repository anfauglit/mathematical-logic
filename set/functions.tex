\documentclass[letterpaper, 10pt]{article}
\usepackage{amsfonts,amsmath,amssymb, amsthm}
\usepackage{mathtools}
\usepackage{enumitem}

\setlength{\parindent}{0pt}

\def\set#1{\left\{#1\right\}}
\newcommand{\tuple}[1]{\langle #1 \rangle}
%\renewcommand{\gcd}[1]{\left( #1 \right)}
\def\abs#1{\left\vert #1 \right\vert}
\renewcommand{\cong}[3][n]{#2 \equiv #3(\textrm{mod}\ #1)}
\newcommand{\divides}[2]{#1 \mid #2}
\chardef\divideOp=`\|
\renewcommand{\vec}[1]{\mathbf #1}
\newcommand{\vect}[1]{\overrightarrow{#1}}
\newcommand{\setcomplement}[1]{\overline{#1}}
\newcommand{\powerset}[1]{\mathcal{P}(#1)}
\newcommand{\entails}{\vdash}
\newcommand{\follows}{\vDash}
\def\myfunc#1#2#3{#1 \colon #2 \to #3}
\let\implies=\rightarrow
\let\iff=\leftrightarrow

% Symbols redefenitions
\let\preorder\preccurlyeq
\let\emptyset\varnothing
\let\le\leqslant
\let\ge\geqslant
\let\epsilon\varepsilon

% Sequence macros

% A regular sequence indexed by integers with a comma as a delimiter for
% terms.

\def\finitesequence#1#2#3#4#5{%
	%1:variable 2:first index 3:last index 4:number of initial terms
	%5:infinity flag
	\newcount\index
	\index=#2
	\advance\index by -1 % After executing the body of the loop the value of index
	% will be equal to the number of terms already printed
	\loop
		\advance\index by 1 
		#1_\number\index,
	\ifnum\index<#4
	\repeat
	\ldots, #1_#3
	\if#5i,\ldots
	\fi
}

% Finite sequence generator
\def\makefiniteseq#1#2#3#4{%
    %1:sequence name and the name of a sequence variable
    %2:first index 3:last index 4:number of initial terms 5:infinity flag
    \expandafter\def\csname seq#1\endcsname{\finitesequence #1#2#3#4n}
}

\def\rseq#1{%1:sequence name 2:sequence variable
    \csname seq#1\endcsname
}
%
%\def\osequence#1#2#3#4#5{%
%        %1:variable 2:first index 3:last index 4:delimiter
%		%5:number of initial terms
%	\newcount\index
%	\index #2
%	\advance\index -1
%	%\newcount\loopindex
%	%\loopindex #5
%	\loop
%		\advance\index 1
%		#1_\the\index#4
%        %\advance\loopindex by-1
%	%\ifnum\loopindex>0
%	\ifnum\index<#5
%	\repeat
%	\cdots#4
%	#1_#3
%}
%
%\def\makesequence#1#2#3#4#5{%
%    %1:name
%    %2:starting index 3:last index 4:delimiter
%    \expandafter\def\csname seq#1\endcsname##1{\osequence ##1#2#3#4#5}
%}
%\def\myseq#1#2{%1:sequence name 2:sequence variable
%    \csname seq#1\endcsname{#2}
%}
%
%\def\sequence#1#2{%1:first element 2:number of elements
%    \newcount\n
%    \newcount\loopindex
%    \loopindex=#2
%    \n=#1
%    \loop
%    \the\n
%    \advance\n by 1
%    \advance\loopindex by -1
%    \ifnum\loopindex>0
%    ,
%    \repeat
%}
%
%\def\dsequence#1#2#3{%1:first element 2:number of elements
%    \newcount\n
%    \newcount\loopindex
%    \loopindex=#2
%    \n=#1
%    \loop
%    \myterm{#3}{\the\n}
%    \if\the\n0
%    ,
%    \fi
%    \advance\n by 1
%    \advance\loopindex by -1
%    \ifnum\loopindex=0
%    \ldots
%    \fi
%    \ifnum\loopindex>0
%    \repeat
%}
%\def\baseterm#1#2{%
%	% 1 := variable
%	% 2 := index
%	#1_#2
%}
%
%\def\modterm#1#2#3{%
%	% 1 := variable
%	% 2 := index
%	% 3 := common term of a sequence
%	\csname #3\endcsname{#1}{#2}
%	}
%
%\def\abso#1#2{%
%	\abs{\baseterm{#1}{#2}}
%	}
%
%\def\base#1#2{%
%	\baseterm{#1}{#2}
%	}
%
%\def\reciprocal#1#2{%
%	\ifcat#21
%	\ifnum#2=1
%	1
%	\else\frac{1}{#2}
%	\fi
%	\else\frac{1}{#2}
%	\fi
%}
%
%\def\rise#1 by#2{%
%	\newcount\i
%	\newcount\n
%	\i=#2
%	\n=#1
%	\loop
%	\advance\i by -1
%	\ifnum\i>0
%	\multiply\n by#1
%	\repeat
%	\number\n
%}
%
%\def\decimal#1#2{%
%	\ifx#2n
%	\frac{1}{10^n}
%	\else
%	\newcount\demon
%	%\demon={\rise10 by2}
%	%\number\demon
%	%\demon=\rise10 by2
%	%\number\demon
%	%\frac{1}{\number\demon}
%	\fi
%}
%
%\def\reciprocalx#1#2{%
%	\ifx#2n
%	\frac{#2}{#2 + 1}
%	\else
%	\newcount\denom
%	\denom=#2
%	\advance\denom by1
%	\frac{#2}{\number\denom}
%	\fi
%}
%
%\def\fterm#1#2{%
%	f_#2(#1)
%	}
%
%\def\absseq#1#2#3#4{%
%	% 1 := variable
%	% 2 := index
%	% 3 := common term of a sequence
%	% 4 := operator
%	\newcount\index
%	\index=#2
%	\newcount\loopindex
%	\loopindex=2
%	\loop
%		\modterm{#1}{\number\index}{#3} #4
%	\advance\index by1
%	\ifnum\loopindex>0
%	\advance\loopindex by-1
%	\repeat
%	\cdots #4 \modterm{#1}{n}{#3} 
%}
%
%\def\dsequence#1#2#3#4{%
%	\newcount\index
%	\index=#2
%	\newcount\lindex
%	\lindex=1
%	\loop
%		\term{\the\index}{#1} #4
%	\advance\index by1
%	\ifnum\lindex>0
%	\advance\lindex by-1
%	\repeat
%	\cdots #4\term #3 #1
%}
%
%\def\term#1#2{%1:variable 2:index
%    #1_{#2}
%}
%
%\def\maketerm#1#2{%1:variable
%    \expandafter\def\csname myterm#2\endcsname##1{#1_##1}}
%    
%\def\myterm#1#2{\csname myterm#1\endcsname#2}
 
\begin{document}

\section{Functions}
	Let $A$ and $B$ be sets. Let $\myfunc fAB$. Then:
	$f$ is a \emph{total function on $A$} if for every $a \in A$ there exists an
	element $b \in B$ such that $f(a) = b$.

	$f$ is a \emph{partial function on $A$} if it is not a total function on
	$A$.

	Let $A$ and $B$ be sets. Let $\myfunc fAB$. Then $f$ maps $A$ \emph{onto}
	$B$ or a \emph{onto function} or a \emph{surjective function} if for every
	element $b$ of its co-domain there exists and an element $a$ in its domain such
	that $f(a) = b$.

	Let $S = \mathbb{N}$ or $S = \mathbb{Z}^+$ and $T$ be any non-empty set.
	Then the function $\myfunc aST$ is called a \emph{sequence}. We call $a(n)$ a
	\emph{term} of the sequence, denoted $a_n$.

	Let $X$ and $Y$ be two sets. A function or a map $f$ from $X$ to $Y$,
	denoted $f\colon X \to Y$, is an assignment of an unique element of $Y$ to
	each element of $X$. The set $X$ is called the domain of $f$, and the set
	$Y$ is called a codomain of $f$.
	
	Since an element $y$ of $Y$ is associated with each element $x \in X$ is unique
	we can designate with a symbol $f(x)$.

	Each function has its own intension, the rule by which it is defined, and
	extension, the set of elements satisfying the specified rule. Two functions
	can be compared by comparing either their intensions or extensions.
	Functions are defined by their extension. Two functions $f$ and $g$ are
	equal, and we write $f = g$, if and only if
	their domains and codomains are equal, and for each $x\in X$ their values
	are equal, that is, $f(x) = g(x)$.

	For any set $X$, the identity function $id_X\colon X \to X$ is defined by
	$id_X(x) = x$ for all $x\in X$. 

	Let $f$ and $g$ be two functions $X\stackrel{f}{\to}Y$ and
	$Y\stackrel{g}{\to}Z$. Define a new function $h$ from $X$ to $Z$ as a
	composition of $g$ and $f$, denoted $g\circ f$ or $gf$ such that for all
	$x\in X$ $g \circ f(x) = g(f(x))$.

	\begin{thm*}
		If $X\stackrel{f}{\to}Y$ then $f\circ id_X = f$ and $id_Y \circ f = f$.
	\end{thm*}

	\begin{thm}
		Let $f\colon X\to Y$, $g\colon Y\to Z$, and $h\colon Z\to W$.
		\begin{enumerate}
			\item The composition of mappings is associative; that is,
				$h\circ(g\circ f) = (h\circ g)\circ f$;
			\item If $f$ and $g$ are both one-to-one, then the mapping $g \circ
				f$ is one-to-one;
			\item If $f$ and $g$ are both onto, then the mapping $g \circ f$ is
				onto;
			\item If $f$ and $g$ are bijective, then so is $g \circ f$.
		\end{enumerate}
	\end{thm}

	A \emph{function} is a relation $F$ such that for each $x$ in $dom(F)$ there is
	only one $y$ such that $xFy$.

	Notation $F(x)$ for the value of the function $F$ at the point $x$ only
	makes sense when $F$ is a function and $x\in dom(F)$.

	A set $R$ is \emph{single-rooted} if and only if for each $y \in ran(R)$
	there is only one $x$ such that $xRy$.

	The \emph{inverse} of $F$ is the set
	\[
		F^{-1} = \set{\tuple{u,v} \mid vFu}
	\].
	The \emph{composition} of $F$ and $G$ is the set
	\[
		F \circ G = \set{\tuple{u,v} \mid \exists t (uGt \land tFv)}
	\].
\end{document}
