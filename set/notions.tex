\documentclass[letterpaper, 10pt]{article}
\usepackage{amsfonts,amsmath,amssymb, amsthm}
\usepackage{mathtools}
\usepackage{mathrsfs}

\setlength{\parindent}{0pt}

\newtheorem{thm}{Theorem}[section]
\newtheorem{cor}[thm]{Corollary}
\newtheorem{lem}[thm]{Lemma}
\newtheorem{define}[thm]{Definition}

\newcommand{\set}[1]{\{#1\}}
\newcommand{\setb}[1]{\{#1\}}
\newcommand{\tuple}[1]{\langle #1 \rangle}
\renewcommand{\gcd}[1]{\left( #1 \right)}
\newcommand{\abs}[1]{\left| #1 \right|}
\renewcommand{\cong}[3]{#1 \equiv #2 (mod\ #3)}
\newcommand{\divides}[2]{#1 \mid #2}
\renewcommand{\vec}[1]{\mathbf #1}
\newcommand{\vect}[1]{\overrightarrow{#1}}

\begin{document}
\section{Introduction}
We define a relation of object belonging to a set, and we write
$x \ in A$ if $a$ is an element of $A$, and $x \not\in A$ if it is not.
Generally, the relation $\in$ is not transitive, that is, if $p \in A$ and $A
\in \mathcal{M}$, it does not follow that $p \in \mathcal{M}$.
Another relation that we define for sets is a subset relation. If all elements
of a set $A$ are also elements of a set $B$, we say that $A$ is a subset of $B$,
and write $A \subseteq B$. The set $B$ is called a superset of $A$, and we write
$B \supseteq A$.
Sets are unordered collections of elements, therefore sets are fully determined
by their elements only. We say that a set is defined by its extension.

If $P(x)$ is some proposition or fomula involving a variable $x$, we shall use
the symbol $\set{x \mid P(x)}$ to denote the set of all objects $x$ for which
the formula $P(x)$ is true. Some authors call this set the truth set or the
extension of $P(x)$.

The family $\mathcal{M}$, a collection of sets, of indexed sets is denoted by
$\set{X_i \mid i \in I}$, briefly $\set{X_i}$, where $i$ is a variable ranging
over the index set $I$.

\end{document}
