\documentclass[letterpaper, 10pt]{article}
\usepackage{amsfonts,amsmath,amssymb, amsthm}
\usepackage{mathtools}
\usepackage{enumitem}

\setlength{\parindent}{0pt}

\newtheorem{thm}{Theorem}[section]
\newtheorem{cor}[thm]{Corollary}
\newtheorem{lem}[thm]{Lemma}
\newtheorem{define}[thm]{Definition}

\newcommand{\set}[1]{\{#1\}}
\newcommand{\tuple}[1]{\langle #1 \rangle}
\renewcommand{\gcd}[1]{\left( #1 \right)}
\newcommand{\abs}[1]{\left| #1 \right|}
\renewcommand{\cong}[3]{#1 \equiv #2 (mod\ #3)}
\newcommand{\divides}[2]{#1 \mid #2}
\renewcommand{\vec}[1]{\mathbf #1}
\newcommand{\vect}[1]{\overrightarrow{#1}}

\begin{document}
Formally, the propositional calculus is the study of certain strings (finite
sequences) of leters, made up of an alphabet of six letters. The first two
letters are called 'not' and 'or', denoted by the symbols $\neg$ and $\lor$,
respectively. They are called propositional connectives. The next two letters
are $p$ and $+$, where the latter is just a suffix for repeated attaching to
$p$, as in $p^{++++}$.
The sentences of propositional calculus are obtained as follows:
\begin{enumerate}
	\item The letter $p$ followed by zero or more symbols $+$ is a sentence.
		Sentences obtained in this way are called propositional variables.
	\item If $x$ is a sentence then the juxtaposition of $($, $\neg$, $x$ and
		$)$ is also
		a sentence.
	\item If $x$ and $y$ are sentences, then the juxtaposition of $($, $x$,
$\lnot$, $y$ and $)$ is also a sentence.  
\end{enumerate}

The set of all variables is the smallest set of finite sequences that contains
the one-termed sequence $p$ and that is closed under the operation of suffixing
$+$. The set of all sentences is the smallest set of finite sequences that
contains all variables and is closed under the operations of prefixing $\neg$
and infixing $\lnot$. Sentences are frequently called well-formed formulas,
abbreviated wff.

\end{document}
