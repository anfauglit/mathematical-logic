\documentclass[letterpaper, 10pt]{article}
\usepackage{amsfonts,amsmath,amssymb,amsthm}
\usepackage{mathtools}
\usepackage{enumitem}
\usepackage[T2A]{fontenc}
\usepackage[utf8x]{inputenc}
\usepackage[russian,english]{babel}

\setlength{\parindent}{0pt}

\theoremstyle{definition}
\newtheorem{thm}{Theorem}[section]
\newtheorem{cor}[thm]{Corollary}
\newtheorem{lem}[thm]{Lemma}
\newtheorem{define}[thm]{Definition}

\documentclass[letterpaper, 10pt]{article}

\setlength{\parindent}{0pt}

\theoremstyle{definition}
\newtheorem*{thm*}{Theorem}
\newtheorem*{define*}{Definition}
\newtheorem*{lem*}{Lemma}
\def\set#1{\left\{#1\right\}}
\newcommand{\tuple}[1]{\langle #1 \rangle}
%\renewcommand{\gcd}[1]{\left( #1 \right)}
\def\abs#1{\left\vert #1 \right\vert}
\renewcommand{\cong}[3][n]{#2 \equiv #3(\textrm{mod}\ #1)}
\newcommand{\divides}[2]{#1 \mid #2}
\chardef\divideOp=`\|
\renewcommand{\vec}[1]{\mathbf #1}
\newcommand{\vect}[1]{\overrightarrow{#1}}
\newcommand{\setcomplement}[1]{\overline{#1}}
\newcommand{\powerset}[1]{\mathcal{P}(#1)}
\newcommand{\entails}{\vdash}
\newcommand{\follows}{\vDash}
\def\myfunc#1#2#3{#1 \colon #2 \to #3}
\let\implies=\rightarrow
\let\iff=\leftrightarrow

% Symbols redefenitions
\let\preorder\preccurlyeq
\let\emptyset\varnothing
\let\le\leqslant
\let\ge\geqslant
\let\epsilon\varepsilon

% Sequence macros

% A regular sequence indexed by integers with a comma as a delimiter for
% terms.

\def\finitesequence#1#2#3#4#5{%
	%1:variable 2:first index 3:last index 4:number of initial terms
	%5:infinity flag
	\newcount\index
	\index=#2
	\advance\index by -1 % After executing the body of the loop the value of index
	% will be equal to the number of terms already printed
	\loop
		\advance\index by 1 
		#1_\number\index,
	\ifnum\index<#4
	\repeat
	\ldots, #1_#3
	\if#5i,\ldots
	\fi
}

% Finite sequence generator
\def\makefiniteseq#1#2#3#4{%
    %1:sequence name and the name of a sequence variable
    %2:first index 3:last index 4:number of initial terms 5:infinity flag
    \expandafter\def\csname seq#1\endcsname{\finitesequence #1#2#3#4n}
}

\def\rseq#1{%1:sequence name 2:sequence variable
    \csname seq#1\endcsname
}
%
%\def\osequence#1#2#3#4#5{%
%        %1:variable 2:first index 3:last index 4:delimiter
%		%5:number of initial terms
%	\newcount\index
%	\index #2
%	\advance\index -1
%	%\newcount\loopindex
%	%\loopindex #5
%	\loop
%		\advance\index 1
%		#1_\the\index#4
%        %\advance\loopindex by-1
%	%\ifnum\loopindex>0
%	\ifnum\index<#5
%	\repeat
%	\cdots#4
%	#1_#3
%}
%
%\def\makesequence#1#2#3#4#5{%
%    %1:name
%    %2:starting index 3:last index 4:delimiter
%    \expandafter\def\csname seq#1\endcsname##1{\osequence ##1#2#3#4#5}
%}
%\def\myseq#1#2{%1:sequence name 2:sequence variable
%    \csname seq#1\endcsname{#2}
%}
%
%\def\sequence#1#2{%1:first element 2:number of elements
%    \newcount\n
%    \newcount\loopindex
%    \loopindex=#2
%    \n=#1
%    \loop
%    \the\n
%    \advance\n by 1
%    \advance\loopindex by -1
%    \ifnum\loopindex>0
%    ,
%    \repeat
%}
%
%\def\dsequence#1#2#3{%1:first element 2:number of elements
%    \newcount\n
%    \newcount\loopindex
%    \loopindex=#2
%    \n=#1
%    \loop
%    \myterm{#3}{\the\n}
%    \if\the\n0
%    ,
%    \fi
%    \advance\n by 1
%    \advance\loopindex by -1
%    \ifnum\loopindex=0
%    \ldots
%    \fi
%    \ifnum\loopindex>0
%    \repeat
%}
%\def\baseterm#1#2{%
%	% 1 := variable
%	% 2 := index
%	#1_#2
%}
%
%\def\modterm#1#2#3{%
%	% 1 := variable
%	% 2 := index
%	% 3 := common term of a sequence
%	\csname #3\endcsname{#1}{#2}
%	}
%
%\def\abso#1#2{%
%	\abs{\baseterm{#1}{#2}}
%	}
%
%\def\base#1#2{%
%	\baseterm{#1}{#2}
%	}
%
%\def\reciprocal#1#2{%
%	\ifcat#21
%	\ifnum#2=1
%	1
%	\else\frac{1}{#2}
%	\fi
%	\else\frac{1}{#2}
%	\fi
%}
%
%\def\rise#1 by#2{%
%	\newcount\i
%	\newcount\n
%	\i=#2
%	\n=#1
%	\loop
%	\advance\i by -1
%	\ifnum\i>0
%	\multiply\n by#1
%	\repeat
%	\number\n
%}
%
%\def\decimal#1#2{%
%	\ifx#2n
%	\frac{1}{10^n}
%	\else
%	\newcount\demon
%	%\demon={\rise10 by2}
%	%\number\demon
%	%\demon=\rise10 by2
%	%\number\demon
%	%\frac{1}{\number\demon}
%	\fi
%}
%
%\def\reciprocalx#1#2{%
%	\ifx#2n
%	\frac{#2}{#2 + 1}
%	\else
%	\newcount\denom
%	\denom=#2
%	\advance\denom by1
%	\frac{#2}{\number\denom}
%	\fi
%}
%
%\def\fterm#1#2{%
%	f_#2(#1)
%	}
%
%\def\absseq#1#2#3#4{%
%	% 1 := variable
%	% 2 := index
%	% 3 := common term of a sequence
%	% 4 := operator
%	\newcount\index
%	\index=#2
%	\newcount\loopindex
%	\loopindex=2
%	\loop
%		\modterm{#1}{\number\index}{#3} #4
%	\advance\index by1
%	\ifnum\loopindex>0
%	\advance\loopindex by-1
%	\repeat
%	\cdots #4 \modterm{#1}{n}{#3} 
%}
%
%\def\dsequence#1#2#3#4{%
%	\newcount\index
%	\index=#2
%	\newcount\lindex
%	\lindex=1
%	\loop
%		\term{\the\index}{#1} #4
%	\advance\index by1
%	\ifnum\lindex>0
%	\advance\lindex by-1
%	\repeat
%	\cdots #4\term #3 #1
%}
%
%\def\term#1#2{%1:variable 2:index
%    #1_{#2}
%}
%
%\def\maketerm#1#2{%1:variable
%    \expandafter\def\csname myterm#2\endcsname##1{#1_##1}}
%    
%\def\myterm#1#2{\csname myterm#1\endcsname#2}


\begin{document}
Formally, the propositional calculus is the study of certain strings (finite
sequences) of letters, made up of an alphabet of six letters. The first two
letters are called 'not' and 'or', denoted by the symbols $\neg$ and $\lor$,
respectively. They are called propositional connectives. The next two letters
are $p$ and $+$, where the latter is just a suffix for repeated attaching to
$p$, as in $p^{++++}$.
The sentences of propositional calculus are obtained as follows:
\begin{enumerate}
	\item The letter $p$ followed by zero or more symbols $+$ is a sentence.
		Sentences obtained in this way are called propositional variables.
	\item If $x$ is a sentence then the juxtaposition of $($, $\neg$, $x$ and
		$)$ is also
		a sentence.
	\item If $x$ and $y$ are sentences, then the juxtaposition of $($, $x$,
$\lnot$, $y$ and $)$ is also a sentence.  
\end{enumerate}

The set of all variables is the smallest set of finite sequences that contains
the one-termed sequence $p$ and that is closed under the operation of suffixing
$+$. The set of all sentences is the smallest set of finite sequences that
contains all variables and is closed under the operations of prefixing $\neg$
and infixing $\lnot$. Sentences are frequently called well-formed formulas,
abbreviated wff.

\section{Propositional Calculus}
\subsection{Formualas}
d(
The propositional connectives are negation ($\neg$), conjucnction ($\land$),
disjunction ($\lor$), implication ($\implies$), biimplication ($\iff$). The
arity of the connectors $\land, \lor, \implies, \iff$ is equal to $2$ and the
arity of $\neg$ is $1$.
d)

d(
A propositional language $L$ is a set of propositional atoms $p,q,r,\ldots$. An
atomic $L$-formula is an atom of $L$.
d)

d(
The set of $L$-formulas is generated inductively according to the following
rules:
\begin{enumerate}
	\item If $p$ is an atomic $L$-formula, then $p$ is an $L$-formula.
	\item If $A$ is an $L$-formula, then $(\neg A)$ is an $L$-formula.
	\item If $A$ and $B$ are $L$-formulas, then $(A \land B)$, $(A \lor B)$, $(A
		\implies B)$, and $(A \iff B)$ are $L$-formulas.
\end{enumerate}
d)

d(
If $A$ is a formula, the degree of $A$ is the number of occurences of
propositional connectives in $A$. This is the same as the number of applications
of rules $2$ and $3$ to generate $A$ from propositional atoms.
d)

The degree of a formula $F$ is a function, which maps each formula to a natural
number.
$$
deg(F) =
\begin{cases}
	0, & \text{if } F \text{ is a propositional atom} \\
	1 + deg(A), & \text{if } F=(\neg A) \text{, where $A$ is a $L$-formula} \\
	1 + deg(A) + deg(B), & \text{if } F = (A \square B) \text{, where $A$ and
	$B$ are $L$-formulas}
\end{cases}
$$

Now we have to prove that every $L$-formula has a unique degree.

\makefiniteseq A1n2
d(
Let $L$ be a propositional language. A formation sequence is a finite sequence
$\rseq A$ such that each term of the sequence is obtained from previous tersm by
application of oen of the rules in the definition of $L$-formula. A formation
sequence for $A$ is a formation sequence whose last term is $A$.
d)

t(
$A$ is an $L$-formula if and only if there exists a formation sequence for $A$.
t)

d(
A formation tree is a finite rooted dyadic tree where each ndoe carries a
formula and each non-atomic formula branches to its immediate subformulas.
d)

t(
Every formula $A$ has a unique formation tree which carries $A$ at its root.
t)

d(
The set of $L$-formulas is generated inductively according to the following
rules:
\begin{enumerate}
	\item If $p$ is an atomic $L$-formula, then $p$ is an $L$-formula.
	\item If $A$ is an $L$-formula, then $(\neg A)$ is an $L$-formula.
	\item If $A$ and $B$ are $L$-formulas and $\square$ is a binary connective,
		then $\square A\, B$ is an $L$-formula.
\end{enumerate}
d)

\subsection{Assignments and Satisfiability}

d(
There is a set with two distinct elements $T$ and $F$ called a truth set.
d)

d(
Let $L$ be a propositional language. An $L$-assignment is a mapping
$$
M \colon \set{p \mid p \text{ is an atomic $L$-formula}} \to \set{T,F}
$$
d)

l(
Given an $L$-assignment $M$, there is a unique $L$-valuation
$$
v_M \colon \set{A \mid A \text{ is an $L$-formula} \to \set{T,F}
$$
given by the following clauses:
l)

You have to prove that for each $L$-formula $A$ has a truth value under
$L$-valuation for every $L$-assignment. And also you have to prove that this
function, $L$-valuation, is unique, i.e. if there is another function $v_M'$
defined as above for every $L$-assignment functions $v_M$ and $v_M'$ will be
equal.

There is an obvious one-to-one correspondence between $L$-assignment and
$L$-valuations.

Fix a propositional language $L$.
d(
Let $M$ be an assignment. A formula $A$ is said to be true under $M$ if $v_M(A)
= T$, and false under $M$ if $v_M(A) = F$.
d)

d(
A set of formulas $S$ is said to be satisfiable if there exists an assignment
$M$ which satisfies $S$, i.e., $v_M(A) = T$ for all $A \in S$.
d)

d(
Let $S$ be a set of formulas. A formula $B$ is said to be a logical consequence
of $S$ if it true under all assignments which satisfy $S$.
d)

d(
A formula $B$ is said to be logically valid (or a tautology) if $B$ is true
under all assignments. Equivalently, $B$ is a logical consequence of the empty
set.
d)

c(
\begin{enumerate}
	\item $B$ is logically valid if and only if $\neg B$ is not satisfiable.
	\item $B$ is satisfiable if and only if $\neg B$ is not logically valid.
	\item $B$ is a logical consequence of $\rseq A$ if and only if $(A_1 \land
		\cdots \land A_n) \implies B$ is logically valid.
	\item $A$ is logically equivalent to $B$ if and only if $A \iff B$ is
		logically valid.
\end{enumerate}
c)

\end{document}
