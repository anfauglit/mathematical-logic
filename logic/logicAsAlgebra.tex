\documentclass[letterpaper, 10pt]{article}
\usepackage{amsfonts,amsmath,amssymb, amsthm}

\setlength{\parindent}{0pt}

\newtheorem{thm}{Theorem}[section]
\newtheorem{cor}[thm]{Corollary}
\newtheorem{lem}[thm]{Lemma}
\newtheorem{define}[thm]{Definition}

\newcommand{\set}[1]{\{#1\}}
\newcommand{\tuple}[2]{\langle #1, #2 \rangle}
\newcommand{\abs}[1]{|#1|}

\begin{document}
	A formal language is an output set of all true sentences of some automaton.
	There are four aspects of a formal language: letters, words, sentences and
	truths. Letters constitute words, words constitute sentences and truths are
	special type of sentences.
\end{document}
