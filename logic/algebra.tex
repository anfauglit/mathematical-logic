\documentclass[letterpaper, 10pt]{article}
\usepackage{amsfonts,amsmath,amssymb, amsthm}
\usepackage{mathtools}
\usepackage{enumitem}

\setlength{\parindent}{0pt}

\newtheorem{thm}{Theorem}[section]
\newtheorem{cor}[thm]{Corollary}
\newtheorem{lem}[thm]{Lemma}
\newtheorem{define}[thm]{Definition}

\newcommand{\set}[1]{\{#1\}}
\newcommand{\tuple}[1]{\langle #1 \rangle}
\renewcommand{\gcd}[1]{\left( #1 \right)}
\newcommand{\abs}[1]{\left| #1 \right|}
\renewcommand{\cong}[3]{#1 \equiv #2 (mod\ #3)}
\newcommand{\divides}[2]{#1 \mid #2}
\renewcommand{\vec}[1]{\mathbf #1}
\newcommand{\vect}[1]{\overrightarrow{#1}}
\newcommand{\comp}[1]{\overline{#1}}
\newcommand{\powerset}[1]{\mathcal{P}(#1)}
\renewcommand{\implies}{\Rightarrow}
\newcommand{\entails}{\vdash}
\newcommand{\bicond}{\Leftrightarrow}

\begin{document}

\section{Language as an algebra}

We can treat negation and disjuction as operations on the set of sentences of
the propositional calculus. There is associated with the propositional calculus
a natural algebraic structure whose universe is the set of all sentences and
whose operations are negation and disjunction.

A finite sequence is a function $x$ whose domain is the set of natural numbers
smaller than some fixed natural number $n$, that is, the set
$\set{0,\ldots,n-1}$. It is customary to denote the value of $x$ at $i$ by
$x_i$, where $0 \leq i < n$. If $n = 0$, then $x$ is the empty function.

There is a natural binary operation on finite sequences. If $x$ and $y$ are
finite sequences, so is $xy$. If $x$ and $y$ are of lengths $m$ and $n$
respectively, then $xy$ is of length $m + n$. The values of $xy$ are given by
\[
	(xy)_i=
	\begin{cases}
		x_i\quad\textrm{if}\ i < m \\
		y_{i - m}\quad\textrm{if}\ i \geq m \\
	\end{cases}
\]
where $0 \leq i < m + n$. The operation $xy$ is called concatenation. The
operation of concatenation is associative. The set of all non-emtpy finite
sequences of $A$ is called the free semigroup generated by $A$. If we add the
empty sequence to the set above, the semigroup acquires a unit.j


\section{Theorems}

\subsection{Axioms of propositional calculus}

We can specify the set of all axioms of the propositional calculus either using
axiom schemata or by substitution rules.

\begin{enumerate}[label=T\arabic*, left=0pt]
	\item $x \lor x \implies x$.
	\item $x \implies x \lor y$.
	\item $x \lor y \implies y \lor x$.
	\item $(x \implies y) \implies (z\lor x \implies z \lor y)$.
\end{enumerate}

Define a binary operation on the set of all sentences, denoted $\mu$ and called
modus ponens, such that, if $x$ and $z$ are sentences and if $z$ has the form $x
\implies y$, then $\mu(x,z) = y$, otherwise $\mu(x,z) = x$.
The sentence $y$ is a theorem if and only if there exists a finite sequence of
sentences, whose last term is $y$, such that each sentence in the sequence is
either an axiom, or the result of applying modus ponens to some pair of
preceding sentences in the sequence. A finite sequence with these properties is
called a proof (a formal proof, a formal proof of $y$).

\section{Formal proofs}

\begin{enumerate}[label=T\arabic*, left=0pt, resume]
	\item $\entails (x \implies y) \implies ((z \implies x) \implies (z \implies
		y))$,
	\item $x \implies x$,
	\item $x \lor \neg x$,
	\item $\entails x \implies x \lor x$,
	\item $\entails x \implies y \lor x$, 
	\item $\entails x \implies \neg \neg x$, 
	\item $\entails \neg \neg x \implies x$, 
	\item $\entails (x \implies y) \implies (\neg y \implies \neg x)$.
\end{enumerate}
\section{Entailment}

Implication is derived binary operation on on the set of sentences of the
propositional logic. It is defined in terms of the primitive operations.

Let us introduce a binary relation, called entailment. If $x$ and $y$ are
sentences, then the relation of entailment, denoted by $x \leq y$, means that $x
\implies y$ is a theorem, that is, $\entails x \implies y$.

\section{Logical equivalence}

We introduce a new binary relation called equivalence (symbolized by $\equiv$).
If $x$ and $y$ are sentences, then
\begin{equation*}
	x \equiv y
\end{equation*}
means, by definition, that $x\ leq y$ and $y \leq x$, that is, that both $x
\implies y$ and $y \implies x$ are theorems.

We shall show that logical equivalence is a congruent relation on the algebra of
sentences, that is, it is compatible with propositional connectives.

\section{Conjunction}

Let us introduce a new propositional connective, as an abbreviation. Let $x$ and
$y$ be sentences, then $x \land y$ means $\neg(\neg x \lor \neg y)$.

Another abbreviation is a bi-implication or biconditional connective. 
Let again $x$ and $y$ be sentences, then $x \bicond y$ means $(x \implies
y)\land (y\implies x)$.

Both conjunction and biconditional are binary operations on the set of all
sentences of the propositional calculus. They are derived operations in the
sense that they are defined in terms of the basic operations of negation and
disjunction. The relation $\equiv$ is a congruence relation with respect to
conjuction and biconditional.
\end{document}
