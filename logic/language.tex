\documentclass[letterpaper, 10pt]{article}
\usepackage{amsfonts,amsmath,amssymb, amsthm}
\usepackage{mathtools}

\setlength{\parindent}{0pt}

\newtheorem{thm}{Theorem}[section]
\newtheorem{cor}[thm]{Corollary}
\newtheorem{lem}[thm]{Lemma}
\newtheorem{define}[thm]{Definition}

\newcommand{\set}[1]{\{#1\}}
\newcommand{\tuple}[1]{\langle #1 \rangle}
\renewcommand{\gcd}[1]{\left( #1 \right)}
\newcommand{\abs}[1]{\left| #1 \right|}
\renewcommand{\cong}[3]{#1 \equiv #2 (mod\ #3)}
\newcommand{\divides}[2]{#1 \mid #2}
\renewcommand{\vec}[1]{\mathbf #1}
\newcommand{\vect}[1]{\overrightarrow{#1}}

\begin{document}
	A first-order language $\mathcal{L}$ is an infinite collection of distinct
	symbols, no one of which is properly contained in another. The collection
	includes the following categories:
	\begin{enumerate}
		\item Parentheses: $($, $)$.
		\item Logical connectives: $\lor$, $\neg$.
		\item Quantifier: $\forall$.
		\item Variables, indexed by positive integers: $v_1,v_2,\ldots,v_n,\ldots$.
		\item Equality symbol: $=$.
		\item Constant symbols: A set of zero or more symbols.
		\item Function symbols: For each positive integer $n$, a set of zero
			or more $n$-ary function symbols.
		\item Relation symbols: For each positive integer $n$, a set of zero or
			more $n$-ary relation symbols.
	\end{enumerate}

	If we allowed some symbols to be proper substrings of other symbols, that
	would make our language ambigous and context-dependent, since there would be
	more than one way to interpret the sequence of symbols of the language.

	There is an $a$ function which maps each symbol from the sets of function
	symbols and relation symbols to a natural number, which indicates the their
	arity.

	We want to be able to refer to objects in the mathematical structures under
	consideration.
	An expression of $\mathcal{L}$ is a finite sequence of symbols of the
	language $\mathcal{L}$. A term of a language is an expression which refers
	to some object of the universe of the discourse.
	If $\mathcal{L}$ is a language, a $\mathcal{L}$-term is a nonempty
	expression of $\mathcal{L}$ such that either:
	\begin{enumerate}
		\item $t$ is a variable, or
		\item $t$ is a constant symbol, or
		\item $t \coloneq ft_{1}t_{2}\ldots t_{n}$ where $f$ is an $n$-ary
			function symbol of $\mathcal{L}$ and each of the $t_i$ is a term of
			$\mathcal{L}$.
	\end{enumerate}
	
	If $\mathcal{L}$ is a language, a $\mathcal{L}$-formula is a nonempty
	expression $\phi$ of $\mathcal{L}$ such that either:
	\begin{enumerate}
		\item $\phi \coloneq =t_{1}t_{2}$, where $t_1$ and $t_2$ are terms of
			$\mathcal{L}$, or
		\item $\phi \coloneq Rt_1t_2\ldots t_{n}$, where $R$ is an $n$-ary
			relation symbol of $\mathcal{L}$ and $t_1, t_2, \ldots, t_n$ are all
			terms of $\mathcal{L}$, or
		\item $\phi \coloneq (\neg\alpha)$, where $\alpha$ is a formula of
			$\mathcal{L}$, or
		\item $\phi \coloneq (\alpha \lor \beta)$, where $\alpha$ and $\beta$
			are formulas of $\mathcal{L}$, or 
		\item $\phi \coloneq (\forall v)(\alpha)$, where $v$ is a variable and
			$\alpha$ is a formula of $\mathcal{L}$.
	\end{enumerate}


\end{document}
