\documentclass[letterpaper, 10pt]{article}
\usepackage{amsfonts,amsmath,amssymb,amsthm}
\usepackage{mathtools}
\usepackage{enumitem}
\usepackage[T2A]{fontenc}
\usepackage[utf8x]{inputenc}
\usepackage[russian,english]{babel}

\setlength{\parindent}{0pt}

\theoremstyle{definition}
% Numbered theorems
\newtheorem{thm}{Theorem}[section]
\newtheorem{cor}[thm]{Corollary}
\newtheorem{lem}[thm]{Lemma}
\newtheorem{define}[thm]{Definition}
% Unnumbered theorems
\newtheorem*{thm*}{Theorem}
\newtheorem*{define*}{Definition}
\newtheorem*{lem*}{Lemma}
\def\set#1{\left\{#1\right\}}
\newcommand{\tuple}[1]{\left< #1 \right>}
%\renewcommand{\gcd}[1]{\left( #1 \right)}
\def\abs#1{\left\vert #1 \right\vert}
\renewcommand{\cong}[3][n]{#2 \equiv #3(\textrm{mod}\ #1)}
\newcommand{\divides}[2]{#1 \mid #2}
\chardef\divideOp=`\|
\renewcommand{\vec}[1]{\mathbf #1}
\newcommand{\vect}[1]{\overrightarrow{#1}}
\newcommand{\setcomplement}[1]{\overline{#1}}
\let\boolneg\setcomplement
\newcommand{\powerset}[1]{\mathcal{P}(#1)}
\newcommand{\entails}{\vdash}
\newcommand{\follows}{\vDash}
\def\myfunc#1#2#3{#1 \colon #2 \to #3}
\let\implies\rightarrow
\let\iff\leftrightarrow
\let\equivx\Longleftrightarrow
\let\impliesx\Longrightarrow

\def\setdiff{\backslash}
\def\forallx#1{\forall #1\,}
\def\existsx#1{\exists #1\,}

% Symbols redefenitions
\let\preorder\preccurlyeq
\let\emptyset\varnothing
\let\le\leqslant
\let\ge\geqslant
\let\epsilon\varepsilon

% Sequence macros

% A regular sequence indexed by integers with a comma as a delimiter for
% terms.

\def\finitesequence#1#2#3#4#5{%
	%1:variable 2:first index 3:last index 4:number of initial terms
	%5:infinity flag
	\newcount\index
	\index=#2
	\advance\index by -1 % After executing the body of the loop the value of index
	% will be equal to the number of terms already printed
	\loop
		\advance\index by 1 
		#1_\number\index,
	\ifnum\index<#4
	\repeat
	\ldots, #1_#3
	\if#5i,\ldots
	\fi
}

% Finite sequence generator
\def\makefiniteseq#1#2#3#4{%
    %1:sequence name and the name of a sequence variable
    %2:first index 3:last index 4:number of initial terms 5:infinity flag
    \expandafter\def\csname seq#1\endcsname{\finitesequence #1#2#3#4n}
}

\def\rseq#1{%1:sequence name 2:sequence variable
    \csname seq#1\endcsname
}
%
%\def\osequence#1#2#3#4#5{%
%        %1:variable 2:first index 3:last index 4:delimiter
%		%5:number of initial terms
%	\newcount\index
%	\index #2
%	\advance\index -1
%	%\newcount\loopindex
%	%\loopindex #5
%	\loop
%		\advance\index 1
%		#1_\the\index#4
%        %\advance\loopindex by-1
%	%\ifnum\loopindex>0
%	\ifnum\index<#5
%	\repeat
%	\cdots#4
%	#1_#3
%}
%
%\def\makesequence#1#2#3#4#5{%
%    %1:name
%    %2:starting index 3:last index 4:delimiter
%    \expandafter\def\csname seq#1\endcsname##1{\osequence ##1#2#3#4#5}
%}
%\def\myseq#1#2{%1:sequence name 2:sequence variable
%    \csname seq#1\endcsname{#2}
%}
%
%\def\sequence#1#2{%1:first element 2:number of elements
%    \newcount\n
%    \newcount\loopindex
%    \loopindex=#2
%    \n=#1
%    \loop
%    \the\n
%    \advance\n by 1
%    \advance\loopindex by -1
%    \ifnum\loopindex>0
%    ,
%    \repeat
%}
%
%\def\dsequence#1#2#3{%1:first element 2:number of elements
%    \newcount\n
%    \newcount\loopindex
%    \loopindex=#2
%    \n=#1
%    \loop
%    \myterm{#3}{\the\n}
%    \if\the\n0
%    ,
%    \fi
%    \advance\n by 1
%    \advance\loopindex by -1
%    \ifnum\loopindex=0
%    \ldots
%    \fi
%    \ifnum\loopindex>0
%    \repeat
%}
%\def\baseterm#1#2{%
%	% 1 := variable
%	% 2 := index
%	#1_#2
%}
%
%\def\modterm#1#2#3{%
%	% 1 := variable
%	% 2 := index
%	% 3 := common term of a sequence
%	\csname #3\endcsname{#1}{#2}
%	}
%
%\def\abso#1#2{%
%	\abs{\baseterm{#1}{#2}}
%	}
%
%\def\base#1#2{%
%	\baseterm{#1}{#2}
%	}
%
%\def\reciprocal#1#2{%
%	\ifcat#21
%	\ifnum#2=1
%	1
%	\else\frac{1}{#2}
%	\fi
%	\else\frac{1}{#2}
%	\fi
%}
%
%\def\rise#1 by#2{%
%	\newcount\i
%	\newcount\n
%	\i=#2
%	\n=#1
%	\loop
%	\advance\i by -1
%	\ifnum\i>0
%	\multiply\n by#1
%	\repeat
%	\number\n
%}
%
%\def\decimal#1#2{%
%	\ifx#2n
%	\frac{1}{10^n}
%	\else
%	\newcount\demon
%	%\demon={\rise10 by2}
%	%\number\demon
%	%\demon=\rise10 by2
%	%\number\demon
%	%\frac{1}{\number\demon}
%	\fi
%}
%
%\def\reciprocalx#1#2{%
%	\ifx#2n
%	\frac{#2}{#2 + 1}
%	\else
%	\newcount\denom
%	\denom=#2
%	\advance\denom by1
%	\frac{#2}{\number\denom}
%	\fi
%}
%
%\def\fterm#1#2{%
%	f_#2(#1)
%	}
%
%\def\absseq#1#2#3#4{%
%	% 1 := variable
%	% 2 := index
%	% 3 := common term of a sequence
%	% 4 := operator
%	\newcount\index
%	\index=#2
%	\newcount\loopindex
%	\loopindex=2
%	\loop
%		\modterm{#1}{\number\index}{#3} #4
%	\advance\index by1
%	\ifnum\loopindex>0
%	\advance\loopindex by-1
%	\repeat
%	\cdots #4 \modterm{#1}{n}{#3} 
%}
%
%\def\dsequence#1#2#3#4{%
%	\newcount\index
%	\index=#2
%	\newcount\lindex
%	\lindex=1
%	\loop
%		\term{\the\index}{#1} #4
%	\advance\index by1
%	\ifnum\lindex>0
%	\advance\lindex by-1
%	\repeat
%	\cdots #4\term #3 #1
%}
%
%\def\term#1#2{%1:variable 2:index
%    #1_{#2}
%}
%
%\def\maketerm#1#2{%1:variable
%    \expandafter\def\csname myterm#2\endcsname##1{#1_##1}}
%    
%\def\myterm#1#2{\csname myterm#1\endcsname#2}


\makefiniteseq a1n2
\makefiniteseq x1n2

\begin{document}
\section{Логика предикатов}
	Не все правильные умозаключения могут быть проанализированные при помощи
	логики высказываний. Примерами таких рассуждениям являются следующие:
	\begin{enumerate}
		\item 3 меньше 5 и 5 меньше 7. Следовательно, 3 меньше 7.
		\item Всякое натуральное число есть целое число, 3 есть натуральное
			число. Следовательно, 3 есть целое число.
		\item Всякое целое число --- рациональное число; $1$ --- целое число;
			следовательно, $1$ --- рациональное число.
		\item Всякий ромб --- параллелограмм; $ABCD$ --- ромб; следовательно,
			$ABCD$ --- параллелограмм.
		\item Плоскость $\alpha$ параллельна плоскости $\beta$ и плоскость
			$\beta$ параллельна плоскости $\gamma$. Следовательно, плоскость
			$\alpha$ параллельна плоскости $\gamma$.
		\item По меньшей мере один студент сдал все экзамены. Следовательно,
			каждый экзамен сдал по крайней мере один студент.
		\item Все друзья Ивана дружат с Петром. Сидор не дружит с Петром. Сидор
			не дружит с Петром.
		\item Простое число два --- четное. Следовательно, существуют простые
			четные числа.
	\end{enumerate}

	В данных примерах истинность заключения обусловленна внутренней структурой
	элементарных высказываний, но логика высказываний рассматривает все
	элементраные высказывания как атомарные, нерасчленяемые на составляющие.
	Требуется расширение логики высказываний до системы, которая позволила бы
	анализировать структуру приведенных выше рассуждений и включала бы в себя логику
	высказываний в качестве составляющей части.

	Традиционная формальная логика расчленяет элементарное высказывание на
	субъект (обычно подлежащее) и предикат (обычно сказуемое), при этом субъект
	может быть дополнением в предложении, а предикат --- определением. Субъект -
	это то, о чем говориться в предложении, а предикат --- то, что утверждается
	или отрицается о субъекте.
	Предикатом называется предложение с одной или несколькими
	переменными, которое превращается в высказывание если все появления
	переменных заменить именами предметов из взятые из соответствующей каждой
	переменной её области значений.
	Примерами таких предложений являются:
	\begin{enumerate}
		\item $x$ --- простое число
		\item $x$ --- четное число
		\item $x$ меньше $y$
		\item $x + y = z$
		\item $x$ --- отец $y$
	\end{enumerate}
	
	Предикат можно рассматривать как высказывательную функцию, а выражение ---
	высказывательной формой, область 
	определения которой есть совокупность всех упорядоченных наборов имен
	предметов выбранных из соответствующих предметных областей, а областью
	значений которой является множество высказываний. Функция которая
	преобразует наборы имен в высказывания.
	Предикат каждому набору значений переменных ставит в соответствие 
	определенное высказывание. Таким образом мы можем говорить о
	предикатах как о функциях, множеством значений которых является
	двухэлементное множество $\set{0,1}$.

	При задании придиката, необходимо указать предметную область на которой он
	задан, т.е. область имена элементов из которой могут быть присвоены
	переменным придиката. Переменных предика еще называются предметными
	переменными. Если все значения переменных, являющихся аргументами
	предиката, принимают значения из одного и того же множества, то такой
	предикат называется однородным.

	Любой предикат разбивает предметную область на два подмножества: множество
	предметов при которых его значение "истина" и при которых --- "ложь". То
	подмножество на котором предикат принимает значение "истина" называется
	облатсью истинности предиката.

	Под $P(x)$ можно понимать как определенный, конкретный предикат, так и
	произвольный. В первом случае $P$ является предикатной постоянной, а во
	втором --- предикатной переменной, значениями которой могут являться
	различные конкретные одноместные предикаты. То подмножество на котором предикат
	принимает значение "истина" называется облатсью истинности предиката.

	Естественным обобщением одноместного предиката есть многоместный предикат, с
	помощью которого выражаются отношения между предметами.
	$n$-местным предикатом $P^n(\rseq x)\colon M^n \mapsto \set{0,1}$ называется
	отображение всех упорядоченных наборов из $n$ предметов, взятых из
	некоторого множества $M$, на двухэлементное множество.  Количество
	переменных предиката часто обозначают в верхнем индексе символа имени
	предиката. Нуль-метсный предикат есть высказывание.

	Для предикатов могут быть определены все логические операции, такие как
	отрицание, конъюнкция, дизъюнкция, импликация и эквиваленция. Результатом
	операции является новый предикат.

	Квантор общности и квантор существования являются унарными операциями
	ставящими в соответствие некотором предикату высказывание. Стоит отметить,
	что предикат не есть высказывание, т.к. по определению о любом высказывании
	мы может заключить истинно оно или ложно, а для предиката такое заключение
	возможно только после подстановки в него вместо переменных имен определенных
	предметов.

	Способы задания предикатов: словестный, при помощи формул, табличный.

	Рассмотрим предикат $P(\rseq x)$, аргументы которого принимают
	значения на множестве $M$. Выберем произвольный набор $(\rseq a)$
	значений переменных. Если предикат примет значение $1$, то говорят что набор
	удовлетворяет предикату $P$, если значение $0$, то говорят что не
	удовлетворяет.

	Пусть предикат $P(\rseq x)$ определен на $M$. Тогда $P$ называется:
	выполнимым, если имеется хотя бы один набор $(\rseq a)$, удовлетворяющий
	предикату $P$;
	тождественно истинным, если всякий набор $(\rseq a)$ удовлетворяет предикату
	$P$;
	тождественно ложным, если ни один набор  $(\rseq a)$ удовлетворяет предикату
	$P$.

	\subsection{Кванторы}
	Пусть $P(x)$ предикат на $M$. Тогда под выражением $\forall P(x)$ понимают
	высказывание истинное тогда и только тогда, когда предикат $P(x)$
	тождественно-истинный и ложное когда существует хотя бы один элемент $x$ не
	удовлятворяющий $P(x)$.
	Символ $\forall$ называют квантором общности. Получившееся высказывание уже
	не зависит от переменной $x$, еще говорят что квантор связывает переменную.

	Пусть даны предикаты $P(\rseq x)$ и $Q(\rseq x)$ на $M$. Предикаты $P$ и $Q$
	назваются равносильными если на любом наборе $(\rseq a)$ они принимают
	одинаковые значения. Предикат $Q$ называется следствием предиката $P$ если
	всякий набор $(\rseq a)$ удовлетворяющий предикату $P$ удовлетворяет и
	предикату $Q$.

	Пусть $P(x)$ некоторый предикат. Под выражением $(\exists x )P(x)$ будем
	понимать высказывание истинное, когда существует элемент множества $M$, для
	которого $P(x)$ истинно, и ложное --- в противном случае, т.е. когда для
	всех элементов множества $M$ предикат $P(x)$ ложен. Символ $\exists x$
	называется квантором существования.

	Навешивание квантора общности или существования на предикат можно
	рассматривать как операцию, которая преобразует предикат в предикат или в
	высказывание. Результатом операции будет высказывание если в получившемся
	выражении будут отсутствовать свободные переменные, в противном случае
	выражение будет предикатом.

	Определим понятие формулы логики предикатов.
	Алфавит логики предикатов состоит из следующих символов:
	\begin{enumerate}
		\item символы предметных переменных
		\item символы предикатов
		\item логические символы
		\item символы кванторов
		\item знаки пунктуации
	\end{enumerate}
\end{document}
