\documentclass[letterpaper, 10pt]{article}
\usepackage{amsfonts,amsmath,amssymb,amsthm}
\usepackage{mathtools}
\usepackage{enumitem}
\usepackage[T2A]{fontenc}
\usepackage[utf8x]{inputenc}
\usepackage[russian,english]{babel}

\setlength{\parindent}{0pt}

\theoremstyle{definition}
% Numbered theorems
\newtheorem{thm}{Теорема}[section]
\newtheorem{cor}[thm]{Следствие}
\newtheorem{lem}[thm]{Лемма}
\newtheorem{define}[thm]{Определение}
% Unnumbered theorems
\newtheorem*{thm*}{Theorem}
\newtheorem*{define*}{Definition}
\newtheorem*{lem*}{Lemma}
\def\set#1{\left\{#1\right\}}
\newcommand{\tuple}[1]{\langle #1 \rangle}
%\renewcommand{\gcd}[1]{\left( #1 \right)}
\def\abs#1{\left\vert #1 \right\vert}
\renewcommand{\cong}[3][n]{#2 \equiv #3(\textrm{mod}\ #1)}
\newcommand{\divides}[2]{#1 \mid #2}
\chardef\divideOp=`\|
\renewcommand{\vec}[1]{\mathbf #1}
\newcommand{\vect}[1]{\overrightarrow{#1}}
\newcommand{\setcomplement}[1]{\overline{#1}}
\newcommand{\powerset}[1]{\mathcal{P}(#1)}
\newcommand{\entails}{\vdash}
\newcommand{\follows}{\vDash}
\def\myfunc#1#2#3{#1 \colon #2 \to #3}
\let\implies=\rightarrow
\let\iff=\leftrightarrow

% Symbols redefenitions
\let\preorder\preccurlyeq
\let\emptyset\varnothing
\let\le\leqslant
\let\ge\geqslant
\let\epsilon\varepsilon

% Sequence macros

% A regular sequence indexed by integers with a comma as a delimiter for
% terms.

\def\finitesequence#1#2#3#4#5{%
	%1:variable 2:first index 3:last index 4:number of initial terms
	%5:infinity flag
	\newcount\index
	\index=#2
	\advance\index by -1 % After executing the body of the loop the value of index
	% will be equal to the number of terms already printed
	\loop
		\advance\index by 1 
		#1_\number\index,
	\ifnum\index<#4
	\repeat
	\ldots, #1_#3
	\if#5i,\ldots
	\fi
}

% Finite sequence generator
\def\makefiniteseq#1#2#3#4{%
    %1:sequence name and the name of a sequence variable
    %2:first index 3:last index 4:number of initial terms 5:infinity flag
    \expandafter\def\csname seq#1\endcsname{\finitesequence #1#2#3#4n}
}

\def\rseq#1{%1:sequence name 2:sequence variable
    \csname seq#1\endcsname
}
%
%\def\osequence#1#2#3#4#5{%
%        %1:variable 2:first index 3:last index 4:delimiter
%		%5:number of initial terms
%	\newcount\index
%	\index #2
%	\advance\index -1
%	%\newcount\loopindex
%	%\loopindex #5
%	\loop
%		\advance\index 1
%		#1_\the\index#4
%        %\advance\loopindex by-1
%	%\ifnum\loopindex>0
%	\ifnum\index<#5
%	\repeat
%	\cdots#4
%	#1_#3
%}
%
%\def\makesequence#1#2#3#4#5{%
%    %1:name
%    %2:starting index 3:last index 4:delimiter
%    \expandafter\def\csname seq#1\endcsname##1{\osequence ##1#2#3#4#5}
%}
%\def\myseq#1#2{%1:sequence name 2:sequence variable
%    \csname seq#1\endcsname{#2}
%}
%
%\def\sequence#1#2{%1:first element 2:number of elements
%    \newcount\n
%    \newcount\loopindex
%    \loopindex=#2
%    \n=#1
%    \loop
%    \the\n
%    \advance\n by 1
%    \advance\loopindex by -1
%    \ifnum\loopindex>0
%    ,
%    \repeat
%}
%
%\def\dsequence#1#2#3{%1:first element 2:number of elements
%    \newcount\n
%    \newcount\loopindex
%    \loopindex=#2
%    \n=#1
%    \loop
%    \myterm{#3}{\the\n}
%    \if\the\n0
%    ,
%    \fi
%    \advance\n by 1
%    \advance\loopindex by -1
%    \ifnum\loopindex=0
%    \ldots
%    \fi
%    \ifnum\loopindex>0
%    \repeat
%}
%\def\baseterm#1#2{%
%	% 1 := variable
%	% 2 := index
%	#1_#2
%}
%
%\def\modterm#1#2#3{%
%	% 1 := variable
%	% 2 := index
%	% 3 := common term of a sequence
%	\csname #3\endcsname{#1}{#2}
%	}
%
%\def\abso#1#2{%
%	\abs{\baseterm{#1}{#2}}
%	}
%
%\def\base#1#2{%
%	\baseterm{#1}{#2}
%	}
%
%\def\reciprocal#1#2{%
%	\ifcat#21
%	\ifnum#2=1
%	1
%	\else\frac{1}{#2}
%	\fi
%	\else\frac{1}{#2}
%	\fi
%}
%
%\def\rise#1 by#2{%
%	\newcount\i
%	\newcount\n
%	\i=#2
%	\n=#1
%	\loop
%	\advance\i by -1
%	\ifnum\i>0
%	\multiply\n by#1
%	\repeat
%	\number\n
%}
%
%\def\decimal#1#2{%
%	\ifx#2n
%	\frac{1}{10^n}
%	\else
%	\newcount\demon
%	%\demon={\rise10 by2}
%	%\number\demon
%	%\demon=\rise10 by2
%	%\number\demon
%	%\frac{1}{\number\demon}
%	\fi
%}
%
%\def\reciprocalx#1#2{%
%	\ifx#2n
%	\frac{#2}{#2 + 1}
%	\else
%	\newcount\denom
%	\denom=#2
%	\advance\denom by1
%	\frac{#2}{\number\denom}
%	\fi
%}
%
%\def\fterm#1#2{%
%	f_#2(#1)
%	}
%
%\def\absseq#1#2#3#4{%
%	% 1 := variable
%	% 2 := index
%	% 3 := common term of a sequence
%	% 4 := operator
%	\newcount\index
%	\index=#2
%	\newcount\loopindex
%	\loopindex=2
%	\loop
%		\modterm{#1}{\number\index}{#3} #4
%	\advance\index by1
%	\ifnum\loopindex>0
%	\advance\loopindex by-1
%	\repeat
%	\cdots #4 \modterm{#1}{n}{#3} 
%}
%
%\def\dsequence#1#2#3#4{%
%	\newcount\index
%	\index=#2
%	\newcount\lindex
%	\lindex=1
%	\loop
%		\term{\the\index}{#1} #4
%	\advance\index by1
%	\ifnum\lindex>0
%	\advance\lindex by-1
%	\repeat
%	\cdots #4\term #3 #1
%}
%
%\def\term#1#2{%1:variable 2:index
%    #1_{#2}
%}
%
%\def\maketerm#1#2{%1:variable
%    \expandafter\def\csname myterm#2\endcsname##1{#1_##1}}
%    
%\def\myterm#1#2{\csname myterm#1\endcsname#2}


\makefiniteseq a1n2
\makefiniteseq x1n2
\makefiniteseq y1m2
\makefiniteseq N1m2
\makefiniteseq M1n2

\begin{document}
\section{Логика предикатов}
	Не все правильные умозаключения могут быть проанализированные при помощи
	логики высказываний. Примерами таких рассуждениям являются следующие:

	\begin{enumerate}
		\item 3 меньше 5 и 5 меньше 7. Следовательно, 3 меньше 7.
		\item Всякое натуральное число есть целое число, 3 есть натуральное
			число. Следовательно, 3 есть целое число.
		\item Всякое целое число --- рациональное число; $1$ --- целое число;
			следовательно, $1$ --- рациональное число.
		\item Всякий ромб --- параллелограмм; $ABCD$ --- ромб; следовательно,
			$ABCD$ --- параллелограмм.
		\item Плоскость $\alpha$ параллельна плоскости $\beta$ и плоскость
			$\beta$ параллельна плоскости $\gamma$. Следовательно, плоскость
			$\alpha$ параллельна плоскости $\gamma$.
		\item По меньшей мере один студент сдал все экзамены. Следовательно,
			каждый экзамен сдал по крайней мере один студент.
		\item Все друзья Ивана дружат с Петром. Сидор не дружит с Петром. Сидор
			не дружит с Петром.
		\item Простое число два --- четное. Следовательно, существуют простые
			четные числа.
	\end{enumerate}

	В данных примерах истинность заключения обусловленна внутренней структурой
	элементарных высказываний, которые логика высказываний рассматривает 
	как нераздельные целые, нерасчленяемые на составляющие. Требуется создание более
	тонкого инструмента для анализа подобных умозаключений.
	Возникает необходимость расширения логики высказываний до логической
	системы, которая позволила бы анализировать структуру приведенных выше
	рассуждений, при этом включала бы в себя логику высказываний. Логика
	предикатов позволяет анализировать в том числе и содержание высказываний, в
	то время, как мы помним, в логике высказываний содержание высказываний не
	рассматривается, рассматриваются исключительно их значения истинности.

	Традиционная формальная логика расчленяет элементарные высказывания на
	субъект (подлежащее) и предикат (сказуемое). Субъект ---
	это то, о чем говориться в предложении, а предикат это то, что утверждается
	или отрицается о субъекте. Однако анализируемые объекты
	могут играть роль дополнений в предложении, а предикат роль определения.

	Предикатом или высказывательной формой или неопределенным высказыванием
	называются выражения с одной или несколькими переменными, которые
	превращаются в конкретные высказывания, если все входящие в них переменные
	заменить предметами выбранными из определенной для каждой переменной области ее
	значений.

	Предикат обобщает понятие высказывания.

	Примерами таких предложений являются:
	\begin{enumerate}
		\item $x$ --- простое число
		\item $x$ --- четное число
		\item $x$ меньше $y$
		\item $x + y = z$
		\item $x$ --- отец $y$
	\end{enumerate}
	
	\begin{define}
		$n$-местным предикатом, определенным на множествах $\rseq M$, называется
		предложение, содержащее $n$ переменных $\rseq x$, превращающееся в
		высказывание при подстановке вместо этих переменных любых конкретных
		элементов из множеств $\rseq M$ соответственно.
	\end{define}

	Для $n$-местного предиката будем использовать обозначение $P(\rseq x)$.
	Переменные $\rseq x$ называют предметными, а элементы множеств $\rseq M$,
	которые эти переменные пробегают --- конкретными предметами. Предикат также
	называют функцией-высказыванием.

	$n$-местный предикат можно рассматривать как высказывательную функцию $n$
	аргументов принимающий значения из множества всех высказываний, т.е. область 
	определения которой есть совокупность всех упорядоченных наборов 
	предметов выбранных из соответствующих предметных областей, а областью
	значений --- множество всех высказываний. Функция есть отображение множества
	таких наборов в множество высказываний.

	Предикат каждому набору значений предметных переменных ставит в соответствие
	определенное высказывание, которое, по определению высказывания, может быть
	либо истинным, либо ложным.  Таким образом мы можем рассматривать $n$
	местный предикат как функцию $n$ аргементов, заданной на множествах $\rseq
	M$ и принимающей значения на двухэлементном множестве $\set{0,1}$. Короче,
	предикатом может называться любая функция отображающая произвольное
	множество в множество $\set{0,1}$.

	При задании придиката, необходимо указать предметную область на которой он
	задан, т.е. область имена элементов из которой могут быть присвоены
	переменным придиката. Переменных предика еще называются предметными
	переменными. Если все значения переменных, являющихся аргументами
	предиката, принимают значения из одного и того же множества, то такой
	предикат называется однородным.

	Любой предикат разбивает предметную область на два подмножества: множество
	предметных наборов для которых его значение есть "истина" и для которых ---
	"ложь". Та часть предметной области на которой предикат принимает значение "истина"
	называется облатсью истинности предиката. Пусть $n$-местный предикат
	$P(\rseq x)$, заданный на множествах $\rseq M$, тогда множеством истинности
	будет множество $P^+ = \set{(\rseq a) \mid \rseq a \in M_1\times M_2\times
	\cdots \times M_n \land \lambda[P(\rseq a)] = 1}$, т.е. множество всех тех
	предметных наборов, которые превращают предикат в истинное высказывание.

	Под $P(x)$ можно понимать как определенный, конкретный предикат, так и
	произвольный. В первом случае $P$ является предикатной постоянной, а во
	втором --- предикатной переменной, значениями которой могут являться
	различные конкретные одноместные предикаты. То подмножество на котором предикат
	принимает значение "истина" называется облатсью истинности предиката.

	Естественным обобщением одноместного предиката есть многоместный
	предикат. Если при помощи однометсных предикатов выражают свойства предметов,
	то отношения между $n$ предметами выражают пользуясь $n$-местными
	предикатами. 

	$n$-местным предикатом $P^n(\rseq x)\colon M^n \mapsto \set{0,1}$ называется
	отображение всех упорядоченных наборов из $n$ предметов, взятых 
	множества $\rseq M$, на двухэлементное множество $\set{0,1}$. Количество
	переменных предиката может быть указано в верхнем индексе имени
	предиката. Нульместный предикат есть высказывание.

	Для предикатов могут быть определены все логические операции, такие как
	отрицание, конъюнкция, дизъюнкция, импликация и эквиваленция. Результатом
	операции является новый предикат.

	Квантор общности и квантор существования являются унарными операциями
	ставящими в соответствие некотором предикату высказывание. Стоит отметить,
	что предикат не есть высказывание, т.к. по определению о любом высказывании
	мы может заключить истинно оно или ложно, а для предиката такое заключение
	возможно только после подстановки в него вместо переменных имен определенных
	предметов.

	Способы задания предикатов: словестный, при помощи формул, табличный.

	\subsubsection{Классификация предикатов}

	Рассмотрим $n$-местный предикат $P(\rseq x)$, аргументы которого принимают
	значения на множестве $M$. Выберем произвольный набор $(\rseq a)$ значений
	аргументов. Если высказывание, получившееся после подстановки вместо
	предметных переменных соответствующих предметов из набора, является
	истинным, то говорят что данный набор удовлетворяет предикату $P(\rseq x)$,
	если же ложным, то говорят что не удовлетворяет.

	Пусть $n$-местный предикат $P(\rseq x)$ определен на множестве $M$. Тогда $P$ называется:
	выполнимым, если имеется хотя бы один набор $(\rseq a)$, удовлетворяющий
	предикату $P$;
	тождественно истинным, если всякий набор $(\rseq a)$ удовлетворяет предикату
	$P$;
	тождественно ложным, если ни один набор  $(\rseq a)$ удовлетворяет предикату
	$P$.

	\begin{define}
		Предикат $P(\rseq x)$, заданный на множествах $\rseq M$, называется:
		\begin{enumerate}
			\item тождественно истинным, если при любой подстановке вместо
				переменных $\rseq x$ любых конкретных предметов $\rseq a$ из
				соответствующих множеств $\rseq M$ он превращается в истинное
				высказывание $P(\rseq a)$;
			\item тождественно ложным, если при любой подстановке вместо
				переменных $\rseq x$ любых конкретных предметов $\rseq a$ из
				соответствующих множеств $\rseq M$ он превращается в ложное
				высказывание $P(\rseq a)$;
			\item выполнимым (опровержимым), если существует по меньшей мере
				один набор конкретных предметов $\rseq a$ из множеств $\rseq M$
				соответственно, при подстановке которого вместо соответствующих
				предметных переменных в предикат $P(\rseq x)$ последний
				превращается в истинное (ложное) высказывание $P(\rseq a)$.
		\end{enumerate}
	\end{define}
	
	\begin{define}
		Множеством истинности предиката $P(\rseq x)$, заданного на множествах
		$\rseq M$, называется совокупность всех упорядоченных $n$ наборов
		$(\rseq a)$, в которых $a_1 \in M_1, a_2 \in M_2, \ldots, a_n \in M_n$,
		таких, что данный предикат обращается в истинное высказывание $P(\rseq
		x)$ при подстановке $x_1 = a_2$, $x_2 = a_2, \ldots, x_n = a_n$. Это
		множество будем обозначать $P^+$.
		\[
			P^+ = \set{(\rseq a) \mid \lambda(P(\rseq a)) = 1}
		\]
	\end{define}

	Отображение множества $n$-местных предикатов, заданных на множествах $\rseq
	M$, в множество всех подмножеств предметной области, называется область
	истинности, обозначается степенью с показателем $+$. Отображение связывает
	каждый $n$-местный предикат с соответствующей ему областью истинности.
	Причем это отображение согласуется с операциями заданными на обоих
	множествах, а именно, следующие равенства выполняются:
	\begin{align*}
		(\lnot P(\rseq x))^+ &= \setcomplement{P^+(\rseq x)} \\
		(P(\rseq x) \land Q(\rseq x))^+ &= P^+(\rseq x) \cap Q^+(\rseq x) \\
		(P(\rseq x) \lor Q(\rseq x))^+ &= P^+(\rseq x) \cup Q^+(\rseq x) \\
	\end{align*}

	Множество $P^+$ истинности $n$-местного предиката $P(\rseq x)$
	представляет собой $n$-арное отношение между элементами множеств $\rseq M$,
	т.е. подмножество прямого (декартого) произведения этих множеств: $P
	\subseteq M_1 \times M_2 \times \cdots \times M_n$.

	\begin{define}
		$n$-местный предикат $P(\rseq x)$, заданный на множествах $\rseq M$,
		будет:
		\begin{enumerate}
			\item тождественно истинным тогда и только тогда, когда $P^+ = М_1
				\times M_2 \times \cdots \times M_n$;
			\item тождественно ложным тогда и только тогда, когда $P^+ = \emptyset$;
			\item выполнимым тогда и только тогда, когда $P^+ \not= \emptyset$;
			\item опровержимым тогда и только тогда, когда $P^+ \not= М_1 \times
				M_2 \times \cdots \times M_n$;
		\end{enumerate}
	\end{define}

	\subsection{Равносильность и следование предикатов}

	\begin{define}
		Два $n$-местных предиката $P(\rseq x)$ и $Q(\rseq x)$, заданных над
		одними и теми же множествами $\rseq M$, называются равносильными, если
		набор предметов $(\rseq a)$ удовлетворяет первому предикату тогда и
		только тогда, когда он увовлетворяет второму.
	\end{define}
	
	Предикаты $P(\rseq x)$ и $Q(\rseq x)$ равносильны тогда и только тогда,
	когда их множества истинности совпадают: $P^+ = Q^+$.
	Отношение равносильности предикатов является отношением эквивалентности.
	Совокупность всех $n$-местных предикатов может быть разбина данным
	отношением на классы эквивалентности, где каждый класс будет содержать в
	себе предикаты определяющие одну и ту же функцию $n$ аргументов, заданную на
	множествах $\rseq M$ и принимающую значения в двухэлементном множестве
	$\set{0;1}$.
	Переход от одного предика к равносильному ему называется равносильным
	преобразованием первого.

	Два предиката могут быть равносильными если их рассматривать над одним
	множеством, и не равносильны, если их рассматривать над другим множеством.
	
	\begin{define}
		Предикат $Q(\rseq x)$, заданный над множествами $\rseq M$, называется
		следствием предиката $P(\rseq x)$, заданного над теми же множествами,
		если все наборы значений предметных переменных удовлетворяющие предикат
		$P(\rseq x)$ удовлетворяют предикат $Q(\rseq x)$.
	\end{define}

	Предикат $Q$ является следствием предиката $P$ тогда и только тогда, когда
	$P^+ \subseteq Q^+$.

	\begin{thm}
		Каждые два тождественно истинных (тождественно ложных) предиката,
		заданных на одних и тех же множествах, равносильны. Обратно, всякий
		предикат, равносильный тождественно истинному (тождественно ложному)
		предикату, сам является тождественно истинным (тождественно ложным)
		предикатом.
	\end{thm}

	\begin{thm}
		Каждый тождественно истинный $n$-местный предикат является следствием
		любого другого $n$-местного предиката, определенного на тех же
		множествах. Каждый $n$-местный предикат является следствием любого
		тождественно ложного $n$-местного предиката, определенного на тех же
		множествах.
	\end{thm}

	\begin{thm}
		Пусть $P(\rseq x)$ и $Q(\rseq x)$ --- два $n$-местных предиката,
		определенных на одних и тех же множествах, такие что $P(\rseq x)
		\Rightarrow Q(\rseq x)$. Тогда:
		\begin{enumerate}
			\item если $P(\rseq x)$ тождественно истинный (выполнимый), то и
				$Q(\rseq x)$ тождественно истинный (выполнимый);
			\item если $Q(\rseq x)$ тождественно ложный (опровержимый), то и
				$P(\rseq x)$ тождественно ложный (опровержимый).
		\end{enumerate}
	\end{thm}
	
	\subsection{Логические операции над предикатами}

	\subsubsection{Отрицание предиката}

	\begin{define}
		Отрицанием $n$-местного предиката $P(\rseq x)$, определенного на
		множествах $\rseq M$, называется новый $n$-местный предикат,
		определенный на тех же множествах, обозначаемый $\lnot P(\rseq x)$,
		который превращается в истинное высказывание при всех тех значениях
		предметных переменных, при которых исходное высказывание превращается в
		ложное высказывание. Новому предикату удовлетворяют все те и только те наборы,
		которые не удовлетворяют исходному.
		\end{define}

	\begin{thm}
		Для $n$-местного предиката $P(\rseq x)$, определенного на множествах
		$\rseq M$, множество истинности его отрицания $\lnot P(\rseq x)$
		совпадает с дополнением множества истинности данного предиката:
		$(\lnot P)^+ = \setcomplement{P^+}$.
	\end{thm}

	\begin{thm}
		Отрицание предиката будет тождественно истинным предикатом тогда и
		только тогда, когда исходный предикат тождественно ложен.
	\end{thm}

	Аналогично алгебре высказываний, которая не делала различия между
	равносильными высказываниями, в алгебре предикатов мы не делается различия
	между равносильными предикатами. Равносильные предикаты отождествляются,
	поэтому можно при рассмотрении использовать любой конкретный предикат из
	множества ему равносильных.

	\subsubsection{Конъюнкция двух предикатов}

	\begin{define}
		Конъюнкцией двух предикатов $P(x)$ и $Q(x)$, заданных на одном и том же
		множестве $M$, называется новый предикат, обозначаемый $P(x) \land
		Q(x)$, который превращается в истинное высказывание при тех и только тех
		значениях предметных переменных $a \in M$, при которых конъюнкция $P(a)\land
		Q(a)$ есть истинное высказывание.
	\end{define}

	\begin{define}
		Конъюнкцией $n$-местного предиката $P(\rseq x)$, определенного на
		множествах $\rseq M$, и $m$-местного предиката $Q(\rseq y)$,
		определенного на множествах $\rseq N$, называется новый $(n +
		m)$-местный предикат, определенный на множествах $\rseq M, \rseq N$,
		обозначаемый $P(\rseq x) \land Q(\rseq y)$, который превращается в
		истинное выскзаывание при всех тех и только тех значениях предметных
		переменных, при которых оба исходных предиката превращаются в истинные
		высказывания.
	\end{define}

	\begin{thm}
		Для двух $n$-местных предикатов $P(\rseq x)$ и $Q(\rseq x)$,
		определенных на одной и той же совокупности множеств $\rseq M$,
		множество истинности конъюнкции $P(\rseq x) \land Q(\rseq x)$ равняется
		пересечению множеств истинности исходных предикатов: $(P \land Q)^+ =
		P^+ \cap Q^+$.
	\end{thm}

	Таким образом, если перед нами стоит задача отыскания множества истинности
	конъюнкции двух предикатов, то решить ее мы можем двумя способами: найти
	область истинности членов конъюнкции по отдельности, а затем взять
	пересечение этих множеств, или непосредственно найти все те наборы из
	предметной области удовлетворяющие новому предикату, т.е. такие наборы
	которые одновременно удовлетворяют каждому члену конъюнкции.

	\begin{cor}
		Конъюнкция двух предикатов тождественно истинна тогда и только тогда, когда
		оба данных предиката тождественно истинны.
	\end{cor}

	\subsubsection{Дизъюнкция двух предикатов}

	\begin{define}
		Дизъюнкцией $n$-местного предиката $P(\rseq x)$, определенного на
		множествах $\rseq M$, и $m$-местного предиката $Q(\rseq y)$,
		определенного на множествах $\rseq N$, называется новый $(n +
		m)$-местный предикат, определенный на множествах $\rseq M, \rseq N$,
		обозначаемый $P(\rseq x) \lor Q(\rseq y)$, который превращается в
		истинное выскзаывание при всех тех и только тех значениях предметных
		переменных, при которых в истинное высказывание превращается по меньшей мере
		один из исходных предикатов.
	\end{define}

	\begin{thm}
		Для $n$-местных предикатов $P(\rseq x)$ и $Q(\rseq x)$, определенных на
		множествах $\rseq M$, множество истинности дизъюнкции $P(\rseq x) \lor
		Q(\rseq x)$ совпадает с объединением множеств истинности исходных
		предикатов: $(P \land Q)^+ = P^+ \cap Q^+$.
	\end{thm}

	\begin{cor}
	Конъюнкция двух предикатов тождественно истинна тогда и только тогда, когда
	оба данных предиката тождественно истинны.
	\end{cor}

	\subsubsection{Импликация двух предикатов}

	\begin{define}
		Импликацией $n$-местного предиката $P(\rseq x)$, определенного на
		множествах $\rseq M$, и $m$-местного предиката $Q(\rseq y)$,
		определенного на множествах $\rseq N$, называется новый $(n +
		m)$-местный предикат, определенный на множествах $\rseq M, \rseq N$,
		обозначаемый $P(\rseq x) \implies Q(\rseq y)$, который превращается в
		истинное выскзаывание при всех тех и только тех значениях предметных
		переменных, при которых в ложное высказывание превращается первый
		предикат или в истинное высказывание превращается второй. 
	\end{define}

	\subsection{Кванторные операции над предикатами}

	Природа предикатов позволяет ввести две кванторные операции или опеарации
	квантификации --- квантор общности и квантор существования. Кванторные
	операции общности и существования можно рассматривать как обобщение операций
	конъюнкции и дизъюнкции для бесконечных областей.
	Еще раз напомним, что значением предиката является предикат, в случае
	нульместного предиката, мы говорим просто высказывание, а значением
	высказывания мы считает его значение истинности.

	\subsubsection{Квантор общности}

	Одноместный предикат может быть превращен в высказывание путем подстановки
	конкретного предмета из предметной области вместо его переменной. Из одноместного предиката можно
	также получить высказывание применив к нему операцию связывания квантором
	общности или существования.

	\begin{define}
		Операцией связывания квантором общности называется правило, по которому
		одноместному предикату $P(x)$, определенному на множестве $M$,
		сопоставляется высказывание, обозначаемое $\forallx xP(x)$, которое
		истинно в том и только том случае, когда предикат $P(x)$ тождественно
		истинен, и ложно в противном случае, т.е.
		\[
			\lambda[(\forall x)(P(x))] = 
			\begin{cases}
				1, & \text{если } P(x) \text{ - тождественно истинный;} \\
				0, & \text{если } P(x) \text{ - опровержимый.}
			\end{cases}
		\]
	\end{define}

	В предикате $P(x)$ переменная $x$ называется свободной, а в высказывании
	$\forallx x P(x)$ она становится связанной.

	Высказывание $(\forall x)(P(x))$ называется универсальным высказыванием для
	предиката $P(x)$. 

	В выражении $(\forall x)(P(x))$ переменная $x$ перестает быть переменной в
	обычном понимании этого слова. В этом случае, такую переменную называют связанной 
	квантором общности или просто связанной.
	Для связанной переменной не имеет смысла замена ее на предмет, выбранный
	из предметной области. Переменные, которые в предикате не являются
	связанными называются свободными.
	Значение предиката зависит от свободных переменных и не зависит от
	переменных являющихся связанными.
	Связанную переменную можно рассматривать уже не как отдельный произвольно
	выбранный предмет из предметной области, а как совокупность всех предметов
	области, или как тип предметов, заданный предметной областью. Например,
	высказывания что все положительные целые числа не меньше единицы,
	записываемое как $(\forall n)(1 \leq n)$, где предикат $1 \leq n$ задан на
	множестве положительных целых чисел, $\mathbb{Z}_+$, можно рассматривать как
	следующее выражение: $1 \leq <\text{положительное целое число}>$, где
	<положительное целое число> есть имя типа. 

	Если одноместный предикат $P(x)$ задан на конечном множестве $M = \set{\rseq
	a}$, то тогда высказывание $(\forall x)(P(x))$ равносильно высказыванию
	$P(a_1) \land P(a_2) \land \ldots \land P(a_n)$.

	\begin{define}
		Операцией связывания квантором общности по переменной $x_1$ называется
		правило, по которому каждому $n$-местному предикату $P(\rseq x)$,
		определенному на множествах $\rseq M$, сопоставляется новый $(n -
		1)$-местный предикат, обозначаемый $(\forall x_1)(P(\rseq x)$, который
		для любых предметов $a_2 \in M_2, \ldots$, $a_n \in M_n$ превращается в
		высказывание $(\forall x_1)(P(x_1, a_2, \ldots, a_n))$, истинное в том и
		только том случае, когда одноместный предикат $P(x_1,a_2,\ldots,x_n)$,
		определенный на множестве $M_1$, тождественно истинен, и ложное в
		противном случае, то есть
	\[
		\lambda[(\forall x_1)(P(x_1,a_2,\ldots,a_n))] = 
		\begin{cases}
			1, & \text{если } P(x_1,a_2,\ldots,a_n) \text{
				- тождественно истинный предикат от} x_1 \\
			0, & \text{если } P(x_1,a_2,\ldots,a_n) \text{ - опровержимый
			предикат от } x_1.
		\end{cases}
	\]
	\end{define}

	\subsubsection{Квантор существования}

	\begin{define}
		Операцией связывания квантором существования называется правило, по которому
		одноместному предикату $P(x)$, определенному на множестве $M$,
		сопоставляется высказывание, обозначаемое $(\exists x)(P(x))$, которое
		истинно в том и только том случае, когда предикат $P(x)$ выполним, и
		ложно в случае если предикат тождественно ложный, т.е.
		\[
			\lambda[(\exists x)(P(x))] = 
			\begin{cases}
				1, & \text{если } P(x) \text{ - выполнимый;} \\
				0, & \text{если } P(x) \text{ - тождественно ложный.}
			\end{cases}
		\]
	\end{define}

	Другими словами, высказывание $\existsx x P(x)$ будет истинно только в том и
	только том случае, когда будет существовать хотя бы один элемент $a \in M$ для
	которого высказывание $P(a)$ истинно.

	Пусть $P(x)$ одноместный предикат заданный на конечном множестве $M =
	\set{\rseq a}$. Если предикат является тождественно истинным, то будут
	истинными все высказывания $P(a_1)$, $P(a_2)$, \ldots, $P(a_n)$ и
	следовательно и конъюнкция этих высказываний. Если же предикат является
	опровержимым, то хотя бы один из членов конъюнкции будет ложен, а
	следовательно и вся конъюнкция. Таким образом приходим к справедливости
	следующей равносильности:
	\[
		\forall x P(x) \equiv P(a_1) \land P(a_2) \land \cdots \land P(a_n)
	\]
	Аналогичным образом можно показать, что справедлива и такая равносильность:
	\[
		\exists P(x) \equiv P(a_1) \lor P(a_2) \lor \cdots \lor P(a_n)
	\]

	Таким образом кванторные операции являются обобщением операций конъюнкции и
	дизъюнкции на случай бесконечный предметных областей.

	Пусть $P(x)$ предикат на $M$. Тогда под выражением $\forall P(x)$ понимают
	высказывание истинное тогда и только тогда, когда предикат $P(x)$
	тождественно истинный и ложное когда существует хотя бы одно значение
	переменной $x$ не удовлятворяющее $P(x)$. т.е. $P(x)$ есть опровержимый
	предикат.

	Символ $\forall$ называют квантором общности. 

	Пусть даны предикаты $P(\rseq x)$ и $Q(\rseq x)$, заданные на $M$. Предикаты
	$P$ и $Q$ назваются равносильными если на любом наборе $(\rseq a)$ они
	принимают одинаковые значения. Предикат $Q$ называется следствием предиката
	$P$ если всякий набор $(\rseq a)$ удовлетворяющий предикату $P$
	удовлетворяет и предикату $Q$.

	Пусть $P(x)$ некоторый предикат. Под выражением $(\exists x )P(x)$ будем
	понимать высказывание истинное, когда существует элемент множества $M$
	удовлетворяющий $P(x)$, и ложное когда для всех элементов множества $M$
	предикат $P(x)$ ложен. Символ $\exists x$ называется квантором
	существования.

	Навешивание квантора общности или существования на предикат можно
	рассматривать как операцию, которая превращает предикат в новый предикат или
	в высказывание. Результатом операции будет высказывание, если результатом
	операции будет выражение в котором отсутствуют свободные переменные, в
	противном случае выражение будет предикатом.

	\subsection{Формулы логики предикатов}

	\subsubsection{Понятие формулы логики предикатов}

	Алфавит логики предикатов включает в себе следующие классы символов:
	\begin{enumerate}
		\item предметные переменные: $x,y,z,x_i,y_i,z_i (i \in \mathbb{N})$;
		\item предметные константы: $a,b,c,a_i,b_i,c_i (i \in mathbb{N})$;
		\item нульместные предикатные переменные (переменные высказывания): $P, Q, R, P_i, Q_i, R_i (i \in
			\mathbb{N})$;
		\item $n$-местные ($n \geq 1$) предикатные переменные: $P(,\ldots,),
			Q(,\ldots,),R(,\ldots,),Q_i(,\ldots,),R_i(,\ldots,) (i \in
			\mathbb{N})$ с указанием числа свободных мест в них;
		\item функциональные символы: $f_i^{n_i}$, $n_i \in \mathbb{N}$, где
			$f_i^{n_i}$ --- $n_i$-местный функциональный символ;
		\item символы логических операций: $\lnot, \land, \lor, \implies, \iff$; 
		\item символы кванторных операций: $\forall, \exists$;
		\item знаки пунктуации: $(,),,$
	\end{enumerate}

	\begin{define}[Формулы логики предикатов]
		\begin{enumerate}
			\item Каждая нульместная предикатная переменная (или переменное
				высказывание) есть формула.
			\item Если $P(,\ldots,)$ --- $n$-местная предикатная переменная, то
				$P(\rseq x)$ есть формула, в которой все предметные переменные
				$\rseq x$ свободны.
			\item Если $F$ --- формула, то $\lnot F$ --- также формула.
				Свободныме (связанные) предметные переменные в формуле $\lnot F$
				те и только те, которые являются свободными (связанными) в $F$.
			\item Если $F_1$ и $F_2$ --- формулы, и если предметные переменные,
				входящие одновременно в обе эти формулы, свободны в каждой из
				них , то выражения $(F_1 \land F_2)$, $(F_1 \lor F_2)$, $(F_1
				\implies F_2)$, $(F_1 \iff F_2)$ также являются формулами. При
				этом предметные переменные, свободные (связанные) хотя бы в
				одной из формул $F_1$, $F_2$ называются свободными (связанными)
				и в новых формулах.
			\item Если $F$ --- формула и $x$ --- предметная переменная, входящая
				в $F$ свободно, то выражения $(\forall x)(F)$ и $(\exists x)(F)$
				также являются формулами, в которых переменная $x$ связанная, а
				все остальные предметные переменные, входящие в формулу $F$
				свободно или связанно, остаются и в новых формулах
				соответственно такими же.
			\item Никаких других формул логики предикатов, кроме получающихся
				согласно пунктам 1--5, нет.
		\end{enumerate}
	\end{define}

	В данном случае мы имеет тройное определение, так как одновременно с
	понятием формула логики предикатов, определяются понятия свободной и
	связанной переменной.

	Формулы, определенные в пунктах $1$ и $2$, называются элементарными (или
	атомарными). Формулы, не являющиеся элементарными, называются составными.

	На основании пунктов 1, 3 и 4 все формулы алгебры высказываний являются
	формулами алгебры предикатов.
	
	В формулах вида $(\forall \xi)(F)$ и $(\exists \xi)(F)$ формула $F$
	называется областью действия квантора $\forall \xi$ или $\exists \xi$
	соответственно. Вхождение (появление) предметной переменной в формулу будет
	связанным, если эта переменная находится в области действия квантора по этой
	переменной.

	Формулы, в которых нет свободных предметных переменных, называются
	замкнутыми, а формулы, содержащие свободные предметные переменные ---
	открытыми.

	\subsection{Значение формулы логики предикатов}

	В логике высказываний, для определения значения формулы, нам нужно было
	задать значения всех пропозициональных переменных, входящих в формулу. В
	логике предикатов, для того чтобы определить значение формулы необходимо
	задать предметную область $M$ и задать значения следующих символов:
	\begin{enumerate}
		\item всех переменных высказываний (нульместных предикатов);
		\item всех свободных предметных переменных;
		\item всех предикатных переменных.
	\end{enumerate}

	Значение формулы определено лишь тогда, когда задана какая-нибудь
	интерпретация входящих в нее символов. Под интерпретацией понимается система
	$M = \tuple{M, f}$, состоящая из непустого множества $M$ и соответствия
	(функции) $f$, сопоставляющего каждому предикатному символу $P$ определенный
	$n$-местный предикат $P(,\ldots,)$. Для данной интерпретации каждая формула
	без свободных переменных представляет собой высказывание, а всякая формула
	содержащая свободные переменные --- предикат заданный на области $M$.

	После замены всех переменных объектами соответствующего типа формула логики
	предикатов превращается в высказывание, имеющее истинное или ложное
	значение.

	Такая замена переменных на конкретные объекты в формулы называется
	интерпретацией формулы.

	\subsection{Классификация формул логики предикатов}

	\begin{define}
		Формула логики предикатов называется выполнимой (опровержимой) на
		множестве $M$, если при некоторой подстановке вместо предикатных
		переменных конкретных предикатов, заданных на этом множестве, она
		превращается в выполнимый (опровержимый) предикат.
	\end{define}

	Чтобы формула была выполнимой (опровержимой) на $M$ должна существовать ее
	истинная (ложная) интерпретация на $M$.

	\begin{define}
		Формула логики предикатов называется тождественно истинной (тождественно
		ложной) на множестве $M$, если при всякой подстановке вместо предикатных
		переменных любых конкретных предикатов, заданных на этом множестве, она
		превращается в тождественно истинный (тождественно ложный) предикат.
	\end{define}

	\begin{define}
		Формула логики предикатов называется общезначимой, или тавтологией
		(тождественно ложной или противоречием), если при всякой подстановке
		вместо предикатных переменных любых конкретных предикатов, заданных на
		каких угодно множествах, она превращается в тождественно истинный
		(тождественно ложный) предикат. Тот факт, что формула $F$ является
		тавтологией, обозначают $\follows F$.
	\end{define}

	\subsection{Тавтологии логики предикатов}
	
	теорема(
	Всякая формула, получающаяся из тавтологии алгебры высказываний заменой
	входящих в нее пропозициональных переменных произвольными предикатными
	переменными, является тавтологией логики предикатов.
	теорема)

	теорема((законы удаления квантора общности и введения квантора
	существования)
	\begin{enumerate}
		\item $\forallx x (P(x)) \implies P(y)$;
		\item $P(y) \implies \existsx x (P(x))$.
	\end{enumerate}
	теорема)
	
	

	\subsection{Равносильные формулы логики предикатов}

	опр(
	Две формулы логики предикатов $A$ и $B$ равносильны в данной интерпретации,
	если на любом наборе значений свободных переменных они принимают одинаковые
	значения, т.е. формулы выражают в данной интерпретации один и тот же
	предикат.
	опр)

	\begin{define}
		Две формулы логики предикатов $A$ и $B$ называются равносильными на
		области $M$, если они принимают одинаковые логические значения при всех
		значениях входящих в них переменных (предикаторных, высказывательных и
		свободных предметных), отнесенных к области $M$.
	\end{define}

	\begin{define}
		Две формулы логики предикатов $A$ и $B$ называются равносильными, если
		они равносильны на всякой области.
	\end{define}

	Равносильность двух формул $A$ и $B$ обозначают $A \equiv B$.
	Все равносильности логики высказываний верны и в логике предикатов. 

	Список основных равносильностей содержащих кванторы.
	\begin{enumerate}
		\item $\lnot (\forallx x A(x)) \equiv \existsx x \lnot (A(x))$
		\item $\lnot (\existsx x A(x)) \equiv \forallx \lnot (A(x))$
		\item $\forallx x A(x) \equiv \lnot (\existsx x \lnot (A(x)))$
		\item $\existsx x A(x) \equiv \lnot (\forallx x \lnot (A(x)))$
		\item $\forallx x A(x) \land \forallx x B(x) \equiv \forallx x(A(x) \land B(x))$
		\item $c \land \forallx x B(x) \equiv \forallx x (c \land B(x))$
		\item $c \lor \forallx x B(x) \equiv \forallx x (c \lor B(x))$
		\item $c \implies \forallx x B(x) \equiv \forallx x (c \implies B(x))$
		\item $\forallx x (B(x) \implies c) \equiv \existsx x B(x) \implies c$
		\item $\existsx x A(x) \lor \existsx x B(x) \equiv \existsx x(A(x) \lor B(x))$
		\item $c \lor \existsx x B(x) \equiv \existsx x (c \lor B(x))$
		\item $c \land \existsx x B(x) \equiv \existsx x (c \land B(x))$
		\item $\existsx x A(x) \land \existsx x B(x) \equiv \existsx x \existsx y (A(x) \land B(y))$
		\item $c \implies \existsx x B(x) \equiv \existsx x (c \implies B(x))$
		\item $\existsx x (B(x) \implies c) \equiv \forallx x B(x) \implies c$
		\item $\existsx x A(x) \land \existsx x B(x) \equiv \existsx x \existsx
			y (A(x) \land B(y))$
		\item $\forallx x A(x) \lor \forallx x B(x) \equiv \forallx x \forallx y
			(A(x) \lor B(y))$
	\end{enumerate}

	теор((Законы коммутативности для кванторов)
	Следующие формулы являются тавтологиями:
	\begin{enumerate}
		\item $\forallx x \forallx y P(x,y) \iff \forallx y \forallx x P(x,y)$
		\item $\existsx x \existsx y P(x,y) \iff \existsx y \existsx x P(x,y)$
		\item $\existsx y \forallx x P(x,y) \implies \forallx x \existsx y
			P(x,y)$
	\end{enumerate}
	теор)

	\subsection{Предваренная нормальная форма. Общезначимость и выполнимость
	формул логики предикатов}

	опр(
	Формула логики предикатов имеет нормальную форму, если она содержит только
	операции конъюнкции, дизъюнкции и кванторные операции, а операция отрицания
	отнесена к элементраным формулам.
	опр)

	теор(
	Для любой формулы существует равносильная ей нормальная форма, причем
	множества свободных и связанных переменных этих формул совпадают.
	теор)

	опр(
	Предваренной нормальной формой формулы логики предикатов называется такая
	нормальная форма, в которой либо полностью отсутствуют кванторные операции,
	либо они используются после всех операций алгебры логики.
	опр)

	теор(
	Всякая формула логики предикатов может быть приведена к равносильной ей
	предваренной нормальной форме.
	теор)

	Игошин - Математическая логика (2016)
	\section{Логика предикатов}
	\subsection{Основные понятия}

	\begin{define}
		$n$-местным предикатом, определенном на множествах $\rseq M$ называется
		предложение, содержащее $n$ переменных $\rseq x$, превращающееся в
		высказывание при подстановке вместо этих переменных любых конкретных
		элементов из множеств $\rseq M$ соответственно.
	\end{define}

	Cunningham - Logical Introduction to Proof (2013)
	\section{Predicate Logic}
	\emph{Predicate} is a statement proclaiming that certain variables satisfy a
	property.
	The \emph{domain} of a predicate is just the collection of allowed values
	for the varibles in the predicate.
	A \emph{universe of discourse} is the set of all objects we are considering
	during our discussion or study.


\end{document}
