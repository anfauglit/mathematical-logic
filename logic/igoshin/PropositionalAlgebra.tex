Предметами алгебры высказываний являются высказывания, способы построения новых
высказываний из уже имеющихся и возникающие при таких построениях
закономерности.
Высказывание есть предложение которое либо истинно, либо ложно.
Предположим существование первоначальной совокупности простейших высказываний,
называемых элементарными или исходными, с известными значениями истинности.
Будем обозначать конкретные высказывания прописными буквами латинского алфавита
взятыми из его начала, такими как $A,B,C,D,\ldots$ или теми же буквами  с
индексами снизу.
Будем обозначать символом $1$ истинное, а символом $0$ ложное высказывания.
Введем функцию $\lambda$ заданную на совокупности всех высказываний и
принимающую значения в двухэлементном множестве $\set{0,1}$. Функция ставит
в соответствие элементарному высказыванию символ $1$, если высказывание истинно,
и $0$ если оно ложно. Функция называется функцией истинности, а значение
функции $\lambda(p)$, где $p$ есть элементарное высказывание, логическим
значением или значением истинности $p$.
Для области значений функции истинности можно было бы выбрать любое двух элементное
множество. Часто также пользуются множества $\set{T,F}$, $\set{true, false}$,
$\set{И, Л}$.

Для того чтобы определить операции на множестве высказываний результатом которых
будут являтся новые высказывания нам будет достаточно определить значения
истинности для таких высказываний. Основной характеристикой любого высказывания
является его значение истинности и без него такой объект не может считаться
высказыванием.

Опишем способы построения составных высказываний из элементраных, другими
словами зададим совокупность операций на множестве высказываний.
Отрицанием высказывания $P$ называется высказывание, обозначаемое $\not P$,
которое истинно, если высказывание $P$ ложно, и ложно, если высказывание $P$
истинно. Такая зависимость логического значения отрицания от логического
значения исходного высказывания согласуется с существующими принципами мышления
человека.
Такое определение позволяет нам одновременно расширять множество высказываний,
добавляя в него составные высказывания, и также расширить функцию истинности,
добавляя в ее область определения новые составные высказывания и определяя для
них логические значения.

Конъюнкцией двух высказываний $P$ и $Q$ называется новое высказывание,
обозначаемое символом $P \wedge Q$, которое истинно лишь в случае, когда истинны
оба исходных высказывания $P$ и $Q$, и ложно во всех остальных случаях.
В табличной форме это будет выглядеть так:
\begin{tabular}{ccc}
A & B & $A\land B$ \\
\hline
0 & 0 & 0 \\
0 & 1 & 0 \\
1 & 0 & 0 \\
1 & 1 & 1 \\
\end{tabular}

Дизъюнкций двух высказываний $P$ и $Q$ называется новое высказывание,
обозначаемое $P\vee Q$, которое истинно в тех случаях, когда хотя бы одно из
высказываний $P$ или $Q$ истинно, и ложно когда оба высказывания ложны.
Импликацией двух высказываний $P$ и $Q$ называется новое высказывание,
обозначаемое $P \rightarrow Q$, которое ложно в единственном случае, когда
высказывание $P$ истинно, а $Q$ - ложно, а во всех остальных случаях - истинно.
Импликация призвана отразить процесс рассуждения, умозаключения. Если мы исходим
из верной посылки и правильно рассуждаем, то мы должны прийти к правильному
выводу, а если мы из верной посылки приходим к неправильному выводу, то это
означает что наше рассуждение было неправильным.
В случае импликации мы хотим чтобы математические теоремы, имеющие форму
импликации, оставались верными и для случаев когда посылка не верна. В таких
случаях теормема не теряет свою верность, но сделать вывод о верности следствия
мы уже не можем.
Эквивалентностью двух высказываний $P$ и $Q$ называется новое высказывание,
обозначаемое $P \leftrightarrow Q$, которое истинно в том и только том случае
когда высказывания $P$ и $Q$ имеют одно и тоже логическое значение, и ложно -
когда их значения разные.
Логические операции на множестве высказываний моделируют зависимости истинности
составных предложений естественного языка от истинности элементарных
высказываний входящих в их состав.
Для того чтобы определить операцию на множестве нужно сперва определить это
множество, т.е. задать те элементы из которых оно состоит. Нельзя определить
операцию на множестве, которое еще полностью не определено. Стоит отметить, что
операция это не машина в которую ты закладываешь один или несколько предметов и
которая преобразует их в какой-то новый предмет. Операция объединяет (ставит в
соответствие) набору предметов один определенный предмет.  Она не создает новые
предметы, она устанавливает связи между уже существующими.

На двухэлементном множестве $\set{0,1}$ мы можем задать пять операций: одну
унарную и четыре бинарных. Таким образом мы определим следующую алгебраическую
структуру $\left<\set{0,1}, \not, \land, \lor, \rightarrow,
\leftrightarrow\right>$. Функцию истинности отображающую множество всех
высказываний в множество $\set{0, 1}$ мы уже задали. Теперь докажем что данная
функция является гомоморфизмом, даже сюръективным гомоморфизмом.

Логические формулы есть схемы построения составных (сложных) высказываний.
Переменные $X,Y,Z$ есть переменные множества элементарных высказываний.

Переменные пробегающие множество высказываний называют пропозициональными
переменными или переменными высказываниями.

Для каждой формулы должна существовать конечная последовательность всех ее
подформул, т.е. такая конечная последовательность которая начинается с входящих
в данную формулу пропозициональных переменных, заканчивается этой формулой, и
каждый член этой последовательности, не являющийся пропозициональной переменной,
есть либо отрицание уже имеющегося члена этой последовательности, либо
получается из двух уже имеющихся членов этой последовательности их соединением с
помощью одного из знаков $\land,\lor,\rightarrow,\leftrightarrow$ и заключением
полученного выражения в скобки.

